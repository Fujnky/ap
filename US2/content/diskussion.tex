\section{Diskussion}
\label{sec:Diskussion}
Wie in Tabelle~\ref{tab:diskb} zu erkennen ist, liegen die mechanisch ermittelten Werte in den meisten Fällen innerhalb des ermittelten Fehlerintervalls. Die Störstellen 1 und 2, die zur Bestimmung des Auflösungsvermögens dienen, können nach Mittelung der beiden Messreihen nicht mehr unterschieden werden. Eine höhere Auflösung ließe sich durch die Nutzung einer Ultraschallsonde mit höherer Frequenz erreichen. Es ist zu beobachten, dass der B-Scan im Gegensatz zum A-Scan deutlich von den mit Messschieber ermittelten Längen abweicht. Da es sich bei dieser Abweichung um eine konstante Verschiebung handelt, ist zu vermuten, dass die Laufzeitkorrektur fehlerhaft gewählt wurde. Da allerdings kein anderer Anhaltspunkt für ebendiese gefunden werden kann, als der \enquote{gelbe Balken} in Abb.~\ref{fig:b1}, ist keine genauere Aussage möglich.
\begin{table}
        \caption{Mittelwerte der Messungen mit A- und B-Scan.}
        \centering
        \label{tab:diskb}
        \begin{tabular}{l@{}S[round-mode=off, table-format=2.0]S[table-format=2.0, round-precision=0, round-mode=places] @{${}\pm{}$} S[table-format=1.0, round-precision=0, round-mode=places] S[table-format=2.0, round-precision=0, round-mode=places] @{${}\pm{}$} S[table-format=1.0, round-precision=0, round-mode=places] S[table-format=2.0, round-precision=0, round-mode=places] @{${}\pm{}$} S[table-format=1.0, round-precision=0, round-mode=places] } \toprule & {$\text{Störstelle}$}& \multicolumn{2}{c}{$d_1/\si{\mm}$}& \multicolumn{2}{c}{$d_2/\si{mm}$}& \multicolumn{2}{c}{$2r/\si{mm}$}\\\midrule& 1 & 16.72125000000000127898 & 1.29674999999999940314 & 58.62674999999999414513 & 2.93475000000000285638 & 4.69200000000001526956 & 3.20847598167728431662 \\
& 2 & 16.65300000000000224532 & 3.27599999999999846878 & 58.42199999999998993871 & 1.36500000000000154543 & 4.96500000000001140421 & 3.54899999999999904432 \\
& 3 & 59.17274999999998641442 & 2.11574999999999935341 & 11.53425000000000011369 & 2.25225000000000097344 & 9.33300000000001084288 & 3.09015017838939343164 \\
& 4 & 51.80174999999999130296 & 2.25225000000000408207 & 19.79249999999999687361 & 2.32049999999999956302 & 8.44575000000000919442 & 3.23378266315162532507 \\
& 5 & 44.36249999999999715783 & 2.32050000000000666844 & 28.05075000000000073896 & 2.38875000000000037303 & 7.62675000000000125056 & 3.33029230136035225840 \\
& 6 & 36.99149999999999494094 & 2.04750000000000387246 & 36.78674999999999783995 & 2.25225000000000408207 & 6.26175000000000370193 & 3.04382757601346964904 \\
& 7 & 28.73324999999999462830 & 2.38875000000000037303 & 44.90850000000000363798 & 2.18399999999999483435 & 6.39825000000000088107 & 3.23666225647656657927 \\
& 8 & 21.08924999999999627676 & 1.97924999999999839950 & 52.68900000000000005684 & 2.45700000000000429168 & 6.26175000000000370193 & 3.15504034245206055331 \\
& 9 & 12.83099999999999951683 & 2.18399999999999927525 & 60.67425000000000068212 & 2.25225000000000763478 & 6.53475000000000605382 & 3.13727366713521371722 \\
& 10 & 4.98224999999999873523 & 2.25224999999999964118 & 69.13724999999999454303 & 1.84275000000000410694 & 5.92050000000000942180 & 2.91004426512725311582 \\
& 11 & 53.71274999999999977263 & 2.11574999999999935341 & 13.03575000000000017053 & 2.38874999999999992895 & 13.29150000000000275691 & 3.19100683562414122463 \\
 \bottomrule \end{tabular} \end{table}


Bezüglich des TM-Scans kann festgestellt werden, dass die Frequenz, mit der das Herzmodell bewegt wurde mit
\begin{equation*}
  \nu_\text{Herz} = \SI{39.1+-0.4}{\per\minute}
\end{equation*}
relativ gering liegt, was allerdings durch die Bedienung des Experimentators zu begründen ist. Schlag- und Herzminutenvolumen scheinen plausibel, allerdings liegen keine Vergleichswerte aus der Literatur vor. Es ist anzumerken, dass bei der Durchführung des Versuchs die maximale Ausbeulung des Herzmodells zu groß gewählt wurde und die Maxima somit teilweise im nicht mehr messbaren Bereich verschwinden, wodurch sich die bestimmung der Tiefen schwieriger gestaltet.
