\section {Aufbau und Durchführung}
\label{sec:durchführung}

Um die verschiedenen Scanverfahren zu betrachten, wird ein Acrylblock mit elf verschiedenen Bohrungen untersucht, siehe Abb.~\ref{fig:block}. Dieser wird zuerst mit einem Messschieber vermessen, daraufhin wird ein A-Scan für jede Störstelle durchgeführt und die ermittelten Tiefen bzw. Zeiten notiert. Dabei wird destilliertes Wasser als Kontaktmittel zwischen Ultraschallkopf und dem Acrylblock genutzt, weil Luft den Ultraschall stark absorbiert. Der Block wird daraufhin umgedreht und die gleichen Messungen noch einmal von der anderen Seite durchgeführt, damit nicht nur der Abstand von der Oberseite, sondern auch der Durchmesser der Löcher bestimmt werden kann. Das Gleiche wird danach noch mit einem B-Scan wiederholt, aus dem dann die jeweiligen Tiefen abgelesen werden können.
\fig{bilder/b.pdf}{Acrylblock mit Störstellen.\cite{anleitungus2}}{block}

Weiterhin wird an einem Herzmodell, bestehend aus einem wassergefüllten Zylinder und einer Membran, ein TM-Scan durchgeführt. Die Membran kann mit einem Gummiball mit Luft befüllt und dadurch ausgebeult werden. Der Ultraschallkopf wird leicht ins Wasser eingetaucht und der Abstand der Membran zum Kopf im Zeitverlauf gemessen, während mit dem Gummiball periodisch Luft in die Membran gepumpt wird. Weiterhin wird ein A-Scan sowohl zum ausgebeulten, als auch zum entspannten Zeitpunkt gemacht. Aus beiden Messungen kann dann das Schlagvolumen bestimmt werden.
Die Frequenz der verwendeten Ultraschallsonde beträgt in beiden Fällen \SI{2}{\mega\hertz}.
