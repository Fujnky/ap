\section{Diskussion}
\label{sec:Diskussion}

Der gemessene Winkel zwischen den brechenden Oberflächen beträgt $\phi=\SI{60.05}{\degree}$. Da das Prisma im Querschnitt ein gleichseitiges Dreieck ist, sind alle Winkel gleich groß. Somit betragen alle drei Winkel $\phi=\SI{60}{\degree}$. Damit weicht der gemessene Winkel um $0.083\%$ vom tatsächlichen Winkel des Prismas ab.
Der Brechungsindex nimmt mit steigender Wellenlänge ab. Es handelt sich wie erwartet um normale Dispersion. Dieser passt zum Einen zu den Messwerten und liefert, wie im Folgenden diskutiert wird, plausible Ergebnisse.   Der Brechungsindex von Flintglas befindet sich im Bereich von 1.5 - 2.0 \cite{chemie.de}. Das heißt, die berechneten Werte sind plausibel.
Für  Flintglas ist die Abbesche Zahl $\nu<50$ \cite{chemie.de}. Also ist der berechnete Wert  $\nu = 21.9 \pm 10$ realistisch.
