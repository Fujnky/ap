\section{Diskussion}
\label{sec:Diskussion}

Über die Genauigkeit der einzelnen Messwerte kann nichts ausgesagt werden, da die Referenzwerte nicht bekannt sind. Als problematisch anzumerken ist, dass bei Kapazitäts- und Induktivitätsmessbrücke teilweise $0$ für $R_2$ bzw. $R_x$ gemessen bzw. berechnet wurde, damit wurde auch der Fehler $0$, da lediglich eine relative Genauigkeit der Referenzbauteile angegeben wurde. Es könnte die Messung negativ beeinflusst haben, dass eins der beiden Potentiometer schwergängig war und teilweise nichtlineares verhalten gezeigt hat. Wie Tabelle \ref{tab:e} zeigt, war die Bestimmung der Resonanzfrequenz ausreichend genau. Wie in Abb. \ref{fig:e1} und \ref{fig:e2} ersichtlich, sind beim Frequenzganz der Wien-Robinson-Brücke teils erhebliche Abweichungen vom Theoriewert vorhanden, diese sind vermutlich dadurch zu erklären, dass die Bauteile für den Messbereich um $\nu_0$ ausgelegt sind und in anderen Frequenzbereichen sich Störeinflüsse ergeben. Aus Tabelle \ref{tab:f} ist weiterhin ersichtlich, dass der Klirrfaktor des Frequenzgenerators sehr gering war, jedoch ist wiederum kein Vergleich mit Referenzwerten möglich, da unbekannt.
