\section{Auswertung}
\label{sec:Auswertung}

\section{Fehlerrechnung}
\label{subsec:fehlerrechnung}

\subsubsection{Mittelwert}
\begin{equation}
\overline{v} = \frac{1}{N} \sum_{i=1}^N v_i
\end{equation}

\subsubsection{Standardabweichung}
\begin{equation}
s_i = \sqrt{\frac{1}{N - 1} \sum_{j=1}^N \left(v_j - \overline{v}\right){^2}}
\end{equation}

wobei $v_j$ ($j = 1, ..., N$) die Messwerte sind.

\subsubsection{Streuung der Mittelwerte}
\begin{equation}
\sigma_i = \frac{s_i}{\sqrt{N}} = \sqrt{\frac{\sum_{j=1}^N \left(v_j - \overline{v_i}\right){^2}}{N \left(N - 1 \right)}}
\end{equation}

\subsubsection{Gaußfehler}
Bei einer fehlerbehafteten Funktion $f$ mit $k$ als fehlerbehafteter Größe und $\sigma_k$ als Ungenauigkeit, gilt:

\begin{equation}
\Delta x_k = \frac{\mathrm{d}f}{\mathrm{d}k}\sigma_k
\end{equation}.

 Der relative Gaußfehler berechnet sich nach:
\begin{equation}
\Delta x_\text{k, rel} = 1 \pm \frac{\Delta x_k}{|x|}\cdot 100\%
\end{equation}.

Der absolute Gaußfehler ergibt sich aus:
\begin{equation}
\Delta x_i = \sqrt{\left(\frac{\mathrm{d}f}{\mathrm{d}k_{1}}\cdot \sigma_{k_{1}}\right)^2 + \left(\frac{\mathrm{d}f}{\mathrm{d}k_{2}}\cdot \sigma_{k_{2}}\right)^2 + ...}
\end{equation}.

\subsection{Wheatstonesche Brücke}
Die Widerstandswerte 10 und 12 wurden mit jeweils 3 verschiedenen Referenzwiderständern $R_2$ nach (\ref{eqn:wheatstone}) vermessen. Die Ergebnisse finden sich in Tabelle \ref{tab:a}.
\subsection{Kapazitätsmessbrücke}
Die Kapazitätswerte 1, 3 und 9 werden mit jeweils 3 verschiendenen Referenzkondensatoren $C_2$ nach (\ref{eqn:kapazität1}) und (\ref{eqn:kapaztät2}) vermessen. Die Ergebnisse finden sich in Tabelle \ref{tab:b1} und Tabelle \ref{tab:b2}.
\subsection{Induktivitätsmessbrücke}
Tabelle \ref{tab:c}
\subsection{Maxwellbrücke}
Tabelle \ref{tab:d}
\subsection{Wien-Robinson-Brücke}
\subsection{Klirrfaktor}

\begin{table}
  \centering
  \caption{Ergebnisse der Widerstandsmessbrücke.}
  \label{tab:a}
  \sisetup{
    round-mode=figures,
    round-precision=3
  }

  \begin{tabular}{
    l@{}
    S[table-format=4.0] @{${}\pm{}$} S[table-format=1.4]
    S[table-format=1.3] @{${}\pm{}$} S[table-format=1.6]
    S[table-format=3.0] @{${}\pm{}$} S[table-format=1.3]
    S[table-format=1.3] @{${}\pm{}$} S[table-format=1.6]
    S[table-format=3.0] @{${}\pm{}$} S[table-format=1.3]}
    \toprule
    \multicolumn{3}{c}{} & \multicolumn{4}{c}{Wert 10} & \multicolumn{4}{c}{Wert 12} \\
    %\midrule
    &\multicolumn{2}{c}{$R_{2} / \Omega$} &
    \multicolumn{2}{c}{$\sfrac{R_3}{R_4}$} &
    \multicolumn{2}{c}{$R_{10} / \Omega$} &
    \multicolumn{2}{c}{$\sfrac{R_3}{R_4}$} &
    \multicolumn{2}{c}{$R_{12} / \Omega$} \\
    \midrule
    \input{build/a1.txt}
    \midrule
    \multicolumn{5}{r}{Mittelwert}\input{build/a2.txt}
    \bottomrule
  \end{tabular}
\end{table}

\begin{table}
  \centering
  \caption{Ergebnisse der Kapazitätsmessbrücke.}
  \label{tab:b1}
  \sisetup{
    zero-decimal-to-integer=true,
    round-mode=figures,
    round-precision=3
  }

  \begin{tabular}{
    l@{}
    S[table-format=3.0] @{${}\pm{}$} S[table-format=1.4]
    S[table-format=3.0] @{${}\pm{}$} S[table-format=2.2]
    S[table-format=1.3] @{${}\pm{}$} S[table-format=1.6]
    S[table-format=3.0] @{${}\pm{}$} S[table-format=2.1]
    S[table-format=3.0] @{${}\pm{}$} S[table-format=1.3]
    }
    \toprule
    %\multicolumn{3}{c}{} & \multicolumn{4}{c}{Wert 10} & \multicolumn{4}{c}{Wert 12} \\
    %\midrule
    &\multicolumn{2}{c}{$C_{2} / \si{\nano\farad}$} &
    \multicolumn{2}{c}{$R_2 / \si{\ohm}$} &
    \multicolumn{2}{c}{$\sfrac{R_3}{R_4}$} &
    \multicolumn{2}{c}{$R_x / \si{\ohm}$} &
    \multicolumn{2}{c}{$C_x / \si{\nano\farad}$} \\
    \midrule
    \multicolumn{11}{c}{Wert 1}\\
    \input{build/b1.txt}
    \midrule
    \multicolumn{11}{c}{Wert 3}\\
    \input{build/b2.txt}
    \midrule
    \multicolumn{11}{c}{Wert 9}\\
    \input{build/b3.txt}
    %\multicolumn{5}{r}{Mittelwert}\input{build/a2.txt}
    \bottomrule
  \end{tabular}
\end{table}
\begin{table}
  \centering
  \caption{Gemittelte Ergebnisse der Kapazitätsmessbrücke.}
  \label{tab:b2}
  \sisetup{
    zero-decimal-to-integer=true,
    round-mode=figures,
    round-precision=3
  }

  \begin{tabular}{
    l@{}
    c
    S[table-format=3.0] @{${}\pm{}$} S[table-format=1.4]
    S[table-format=3.0] @{${}\pm{}$} S[table-format=2.2]
    }
    \toprule
    &&
    \multicolumn{2}{c}{$R_x / \si{\ohm}$} &
    \multicolumn{2}{c}{$C_x / \si{\nano\farad}$} \\
    \midrule
    \input{build/b4.txt}
    \bottomrule
  \end{tabular}
\end{table}

\begin{table}
  \centering
  \caption{Ergebnisse der Induktivitätsmessbrücke.}
  \label{tab:c}
  \sisetup{
    round-mode=figures,
    round-precision=3
  }

  \begin{tabular}{
    l@{}
    S[table-format=2.1] @{${}\pm{}$} S[table-format=1.5]
    S[table-format=2.1] @{${}\pm{}$} S[table-format=1.2]
    S[table-format=1.2] @{${}\pm{}$} S[table-format=1.5]
    S[table-format=3.0] @{${}\pm{}$} S[table-format=2.1]
    S[table-format=3.0] @{${}\pm{}$} S[table-format=1.4]
    }
    \toprule
    &\multicolumn{2}{c}{$L_{2} / \si{\micro\henry}$} &
    \multicolumn{2}{c}{$R_2 / \si{\ohm}$} &
    \multicolumn{2}{c}{$\sfrac{R_3}{R_4}$} &
    \multicolumn{2}{c}{$R_{16} / \si{\ohm}$} &
    \multicolumn{2}{c}{$L_{16} / \si{\micro\henry}$} \\
    \midrule
    \input{build/c1.txt}
    \midrule
    \multicolumn{7}{r}{Mittelwert} \input{build/c2.txt}
    \bottomrule
  \end{tabular}
\end{table}

\begin{table}
  \centering
  \caption{Ergebnisse der Maxwellbrücke.}
  \label{tab:d}
  \sisetup{
    round-mode=figures,
    round-precision=3
  }

  \begin{tabular}{
    l@{}
    S[table-format=3.0] @{${}\pm{}$} S[table-format=1.4]
    S[table-format=4.0] @{${}\pm{}$} S[table-format=2.2]
    S[table-format=3.0] @{${}\pm{}$} S[table-format=2.2]
    S[table-format=3.0] @{${}\pm{}$} S[table-format=1.2]
    S[table-format=3.0] @{${}\pm{}$} S[table-format=2.1]
    S[table-format=3.0] @{${}\pm{}$} S[table-format=1.2]
    }
    \toprule
    &\multicolumn{2}{c}{$C_{4} / \si{\nano\farad}$} &
    \multicolumn{2}{c}{$R_2 / \si{\ohm}$} &
    \multicolumn{2}{c}{$R_3$} &
    \multicolumn{2}{c}{$R_4 / \si{\ohm}$} &
    \multicolumn{2}{c}{$R_{16} / \si{\ohm}$} &
    \multicolumn{2}{c}{$L_{16} / \si{\micro\henry}$} \\
    \midrule
    & 12.46 & 82.03 \\
& 12.47 & 81.84 \\
& 12.31 & 80.21 \\
& 12.47 & 80.34 \\
& 12.28 & 81.13 \\
& 12.41 & 81.13 \\
& 12.35 & 80.29 \\
& 12.13 & 80.29 \\
& 12.35 & 81.13 \\
& 12.50 & 81.06 \\

    \midrule
    \multicolumn{9}{r}{Mittelwert} \input{build/d2.txt}
    \bottomrule
  \end{tabular}
\end{table}

\begin{table}
  \centering
  \caption{Resonanzfrequenz der Wien-Robinson-Brücke.}
  \label{tab:e}
  \sisetup{
    round-mode=figures,
    round-precision=3
  }

  \begin{tabular}{
    l@{}
    S[table-format=3.0] @{${}\pm{}$} S[table-format=1.3]
    S[table-format=3.0] @{${}\pm{}$} S[table-format=1.2]
    S[table-format=4.0]
    S[table-format=4.0] @{${}\pm{}$} S[table-format=1.2]
    S[table-format=1.3] @{${}\pm{}$} S[table-format=1.5]}
    \toprule
    &\multicolumn{2}{c}{$R / \si{\ohm}$} &
    \multicolumn{2}{c}{$C_3 / \si{\nano\farad}$} &
    {$\nu_0^\mathrm{real} / \si{\hertz}$} &
    \multicolumn{2}{c}{$\nu_0^\mathrm{ideal} / \si{\hertz}$} &
    \multicolumn{2}{c}{$\nu_0^\mathrm{real}/\nu_0^\mathrm{ideal}$} \\
    \midrule
    \begin{table}
        \caption{Messdaten, Hysteresekurve.}
        \centering
        \label{de}
        \begin{tabular}{l@{}cc|cc|cc} \toprule & {$I/\si{A}$}& {$B/\si{mT}$}& {$I/\si{A}$}& {$B/\si{mT}$}& {$I/\si{A}$}& {$B/\si{mT}$}\\\midrule& 0,0 & 27,05 & 3,5 & 792,5 & 3,0 & 730,1 \\
& 0,5 & 125,2 & 4,0 & 888,8 & 2,5 & 623,7 \\
& 1,0 & 242,6 & 4,5 & 978,0 & 2,0 & 508,0 \\
& 1,5 & 364,7 & 5,0 & 1059 & 1,5 & 394,5 \\
& 2,0 & 480,2 & 4,5 & 1000 & 1,0 & 274,3 \\
& 2,5 & 590,5 & 4,0 & 921,8 & 0,5 & 150,4 \\
& 3,0 & 692,7 & 3,5 & 829,6 & 0,0 & 27,67 \\
 \bottomrule \end{tabular} \end{table}

    \bottomrule
  \end{tabular}
\end{table}

\begin{figure}
  \centering
  \includegraphics{build/e1.pdf}
  \caption{Frequenzabhängigkeit der Wien-Robinson-Brücke.}
  \label{fig:e1}
\end{figure}

\begin{figure}
  \centering
  \includegraphics{build/e2.pdf}
  \caption{Vergrößerung von Abb. \ref{fig:e1} in der Nähe der Resonanzfrequenz.}
  \label{fig:e2}
\end{figure}

\subsection{Messwerte}

\begin{table}
  \centering
  \caption{Messwerte der Wheatstone-Brücke.}
  \label{tab:a_mess}
  \sisetup{
    round-mode=off,
    round-precision=3
  }

  \begin{tabular}{
    l@{}
    S[table-format=4.1]
    S[table-format=3.1]
    S[table-format=3.1]
    S[table-format=3.1]
    S[table-format=3.1]}
    \toprule
    &&\multicolumn{2}{c}{Wert 10} & \multicolumn{2}{c}{Wert 12} \\
    &{$R_2 / \si{\ohm}$} &
    {$R_3 / \si{\ohm}$} &
    {$R_4 / \si{\ohm}$} &
    {$R_3 / \si{\ohm}$} &
    {$R_4 / \si{\ohm}$} \\
    \midrule
    \input{build/a_mess.txt}
    \bottomrule
  \end{tabular}
\end{table}

\begin{table}
  \centering
  \caption{Messwerte der Kapazitätsmessbrücke.}
  \label{tab:b_mess1}
  \sisetup{
    round-mode=off,
    round-precision=3
  }

  \begin{tabular}{
    l@{}
    S[table-format=3.1]
    S[table-format=3.1]
    S[table-format=3.1]
    S[table-format=3.1]
    S[table-format=3.1]
    S[table-format=3.1]
    S[table-format=3.1]
    S[table-format=3.1]}
    \toprule
    &\multicolumn{4}{c}{Wert 1} & \multicolumn{4}{c}{Wert 3} \\
    &{$C_2 / \si{\nano\farad}$} &
    {$R_2 / \si{\ohm}$} &
    {$R_3 / \si{\ohm}$} &
    {$R_4 / \si{\ohm}$} &
    {$C_2 / \si{\nano\farad}$} &
    {$R_2 / \si{\ohm}$} &
    {$R_3 / \si{\ohm}$} &
    {$R_4 / \si{\ohm}$} \\
    \midrule
    \input{build/b_mess1.txt}
    \bottomrule
  \end{tabular}
%\end{table}

%\begin{table}
%  \centering
%  \caption{Messwerte der Kapazitätsmessbrücke.}
%  \label{tab:b_mess2}
%  \sisetup{
%    round-mode=off,
%    round-precision=3
%  }

  \begin{tabular}{
    l@{}
    S[table-format=3.1]
    S[table-format=3.1]
    S[table-format=3.1]
    S[table-format=3.1]}
    %\toprule
    &\multicolumn{4}{c}{Wert 9} \\
    &{$C_2 / \si{\nano\farad}$} &
    {$R_2 / \si{\ohm}$} &
    {$R_3 / \si{\ohm}$} &
    {$R_4 / \si{\ohm}$} \\
    \midrule
    \input{build/b_mess2.txt}
    \bottomrule
  \end{tabular}
\end{table}

\begin{table}
  \centering
  \caption{Messwerte der Induktivitätsmessbrücke.}
  \label{tab:c_mess}
  \sisetup{
    round-mode=off,
    round-precision=3
  }

  \begin{tabular}{
    l@{}
    S[table-format=2.1]
    S[table-format=2.1]
    S[table-format=3.1]
    S[table-format=3.1]}
    \toprule
    & \multicolumn{4}{c}{Wert 16} \\
    &{$L_2 / \si{\milli\henry}$} &
    {$R_2 / \si{\ohm}$} &
    {$R_3/ \si{\ohm}$} &
    {$R_4 / \si{\ohm}$} \\
    \midrule
    \input{build/c_mess.txt}
    \bottomrule
  \end{tabular}
\end{table}

\begin{table}
  \centering
  \caption{Messwerte der Maxwellbrücke.}
  \label{tab:d_mess}
  \sisetup{
    round-mode=off,
    round-precision=3
  }

  \begin{tabular}{
    l@{}
    S[table-format=3.1]
    S[table-format=4.1]
    S[table-format=3.1]
    S[table-format=3.1]}
    \toprule
    & \multicolumn{4}{c}{Wert 16} \\
    &{$C_4 / \si{\nano\farad}$} &
    {$R_2 / \si{\ohm}$} &
    {$R_3/ \si{\ohm}$} &
    {$R_4 / \si{\ohm}$} \\
    \midrule
    \input{build/d_mess.txt}
    \bottomrule
  \end{tabular}
\end{table}

\begin{table}
  \centering
  \caption{Messwerte der Wien-Robinson-Brücke.}
  \label{tab:e_mess}
  \sisetup{
    round-mode=off,
    round-precision=3
  }

  \begin{tabular}{
    l@{}
    S[table-format=4.0]
    S[table-format=3.1]|
    S[table-format=5.0]
    S[table-format=3.1]}
    \toprule
    &{$\nu / \si{\hertz}$} &
    {$U_\mathrm{Br} / \si{\volt}$}
    &{$\nu / \si{\hertz}$} &
    {$U_\mathrm{Br} / \si{\volt}$}\\
    \midrule
    \input{build/e_mess.txt}
    \bottomrule
  \end{tabular}
\end{table}
