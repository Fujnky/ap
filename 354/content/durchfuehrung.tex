\section{Durchführung}
\label{sec:durchführung}

Zu Beginn werden die Werte der Bauteile, also die Induktivität der Spule, die Kapazität des Kondensators und die Widerstände notiert.

\fig{content/amplitude}{Schaltung zur Untersuchung der Zeitabhängigkeit der Amplitude einer gedämpften Schwingung.}{amplitude}

Um die Zeitabhängigkeit der Amplitude zu untersuchen, wird die Schlatung aus \ref{fig:amplitude} aufgebaut. Der Abstand zwischen zwei vom Nadelimpulsgenerator erzeugten Pulsen, wird so eingestellt, dass die Amplitude der Kondensatorspannung ungefähr um den Faktor 3 bis 8 abnimmt. Am Oszilloskop wird die Spannung in Abhängigkeit der Zeit aufgetragen. Aus dem aufgenommenen Foto sollen der effektive Dämpfungwiderstand $R_mathrm{eff}$ sowie die Abklingdauer $T_\mathrm{ex}$ bestimmt werden.

\fig{content/grenzwiderstand}{Schaltung zur Bestimmung des aperiodischen Grenzwiderstandes $R_\mathrm{ap}$.}{grenzwiderstand}

Zur Bestimmung des aperiodischen Grenzwiderstandes dient die Schaltung aus \ref{fig: grenzwiderstand}. Zunächst wird der variable Widerstand auf seinen Maximalwert eingestellt. Wird nun am Oszilloskop erneut die Spannung gegen die Zeit aufgetragen, ist ein stetiges Abfallen der Spannung zu sehen. Nun wird der Widerstand verringert. ist auf dem Bildschirm des Oszilloskopes ein Überschwingen der Spannung sichtbar, ist der Grenzwiderstand unterschritten und $R$ muss wieder erhöht werden, sodass das Überschwingen gerade nicht mehr sichtbar ist.

\fig{content/phase}{Schaltung eines Schwingkreises zur Untersuchung der Kondensatorspannung und der Phasenverschiebung zwischen Erreger- und Kondensatorspannung.}{phase}

Die Schaltung wird entsprechend \ref{fig:phase} aufgebaut, um die Frequenzabhängigkeit der Kondensatorspannung, sowie die Phasenverschiebung ziwschen Erreger- und Kondensatorspannung zu untersuchen. Am Sinusgenerator wird die Frequenz im Bereich von (...) \si{\Hz} variiert und die Werte der Spannung und der Phasenverschiebung werden notiert. Um die Phasenverschiebung zu ermitteln, sollen beide Spannungsverläufe entsprechend \ref{fig: phasenverschiebung} auf dem Oszilloskop angezeigt werden. Der Abstand $a$ wird ausgemessen, die Periodenlänge $b$ errechnet sich aus $b = 2\pi f$ und die Phasenverschiebung wird mit $\phi = \frac{a}{b} \cdot 360$ oder $\phi = \frac{a}{b} \cdot 2\pi $ berechnet.

\fig{content/phasenverschiebung}{Messung der Phasenverschiebung zwischen zwei Spannungen mit dem Oszillloskop.}{phasenverschiebung}
