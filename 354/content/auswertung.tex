\section{Auswertung}
\label{sec:Auswertung}
Die Gerätedaten der verwendeten Messapparatur lauten
\begin {align}
  L &= \SI{10.11+-0.03}{mH}\\
  C &= \SI{2.098+-0.006}{nF}\\
  R_1 &= \SI{30.1+-0.1}{\ohm}\\
  R_2 &= \SI{509.5+-0.5}{\ohm}\\
  R_\text{Generator} &= \SI{50}{\ohm}
\end{align}

\subsection{Effektiver Dämpfungswiderstand}
\label{sec:a}
Zuerst wird der effektive Dämpfungswiderstand eines $RLC$-Kreises bestimmt. Der gemessene Spannungsverlauf ist zusammen mit oberer und unterer Einhüllenden in Abb.~\ref{fig:a} dargestellt. Nach
\begin{align}
  I(t) &= A_0 e^{-2\pi\mu t} \cos{(2\pi\nu t + \eta)}\\
  \intertext{und}
  T_\text{ex} :&= \frac{1}{2\pi\mu}
\end{align}
lässt sich mit dem durch nichtlineare Ausgleichsrechnung an die Funktion
\begin{equation}
  U_\text{env} = U_0 \cdot \symup{e}^{-\tau \cdot t}
\end{equation}
ermittelten Exponenten der $\symup{e}$-Funktion über
\begin{equation}
  T_\text{ex} = \frac{1}{\tau}
\end{equation}
die Abklingdauer
\begin{equation}
  T_\text{ex} = \SI{183.7+-2.0}{\second}
\end{equation}
bestimmen. Daraus folgt mit
\begin{align}
  T_\text{ex} &= \frac{2L}{R} \\
  R_\text{eff} &= \SI {110.1+-1.2}{\ohm}
\end{align}
für den effektiven Widerstand. Dies ergibt zur Summe aus $R_1$ und $R_\text{Generator}$ eine relative Abweichung von
\begin{equation}
  \Delta R_\text{eff} = \SI{12,2}{\%}.
\end{equation}
\fig{build/a.pdf}{Gedämpfte Schwingung in einem $RLC$-Kreis, mit Einhüllenden.}{a}

\subsection{Aperiodischer Grenzfall}
Um $R_\text{ap}$ zu bestimmen wird ein veränderlicher Widerstand so eingestellt, dass auf dem Oszilloskopschirm der aperiodische Grenzfall zu sehen ist. Der gemessene Widerstand beträgt
\begin{equation}
  R_\text{ap} = \SI{3280}{\ohm}.
\end{equation}
Im Vergleich dazu liegt der theoretische Wert bei
\begin{equation}
  R_\text{ap}^\text{th} = \SI{4390.4+-9}{\ohm},
\end{equation}
hat demnach eine relative Abweichung von
\begin{equation}
  \Delta R_\text{ap} = \SI{33,9}{\%}
\end{equation}

\subsection{Erzwungene Schwingungen}
\subsubsection{Frequenzabhängigkeit der Amplitude}
Die Amplitude eines erregten Schwingkreises ist nach (\ref{eqn:jfjfjfjf}) abhängig von der Erregerfrequenz. In Abb.~\ref{fig:c1} ist die gemessene Amplitude über die Frequenz logarithmisch aufgetragen. In Abb~\ref{fig:c2} ist der Resonanzbereich noch einmal vergrößert linear dargestellt. Die aus den Messdaten bestimmten Werte sind
\begin{align}
  q &= \num{4.34+-0.01} \\
  \nu_+ - \nu_- &= \SI{8141.01}{\hertz}
\end{align}
für Resonanzüberhöhung und Resonanzkurvenbreite. Die zugehörigen theoretischen Werte lauten
\begin{align}
  q_\text{th} &= \num{3.92+-0.009},\\
  \nu_+ - \nu_- &= \SI{8810+-30}{\hertz}.
\end{align}
Damit liegen die relativen Abweichungen bei
\begin{align}
  \Delta q &= \SI{10.72}{\%} \\
  \Delta (\nu_+ - \nu_-) &= \SI{8.19}{\%}.
\end{align}

\fig{build/c.pdf}{Frequenzabhängigkeit der Amplitude des erregten $RLC$-Kreises, doppeltlogarithmische Darstellung.}{c1}
\fig{build/c_linear.pdf}{Frequenzabhängigkeit der Amplitude des erregten $RLC$-Kreises, lineare Darstellung im Resonanzbereich.}{c2}

\subsubsection{Frequenzabhängigkeit der Phasenverschiebung}
Nicht nur die Amplitude sondern auch die Phasenverschiebung der im $RLC$-Kreis ist eine Funktion der Frequenz. Analog zum vorherigen Abschnitt finden sich in den Abbildungen \ref{fig:d1} und \ref{fig:d2} Plots der gemessenen Phasenverschiebung. Die aus den Messdaten bestimmten Werte sind
\begin{align}
  \nu_\text{res} &= \SI{34850+-98}{\hertz} \\
  \nu_1 &= \SI{38400+-180}{\hertz}\\
  \nu_2 &= \SI{31290+-95}{\hertz}
\end{align}
für Resonanzüberhöhung und Resonankurvenbreite. Die zugehörigen theoretischen Werte lauten
\begin{align}
  \nu_\text{res}^\text{th} &= \SI{34560+-70}{\hertz} \\
  \nu_1^\text{th} &= \SI{39241+-80}{\hertz}\\
  \nu_2^\text{th} &= \SI{30433+-60}{\hertz}
\end{align}
Demnach liegen die relativen Abweichungen bei
\begin{align}
  \Delta \nu_\text{res} &= \SI{0.074}{\%} \\
  \Delta \nu_1 &= \SI{0.45}{\%}\\
  \Delta \nu_2 &= \SI{0.26}{\%}
\end{align}

\fig{build/d.pdf}{Frequenzabhängigkeit der Phasenverschiebung des erregten $RLC$-Kreises, einfachlogarithmische Darstellung.}{d1}
\fig{build/d_linear.pdf}{Frequenzabhängigkeit der Phasenverschiebung des erregten $RLC$-Kreises, lineare Darstellung im Resonanzbereich.}{d2}

\subsection{Messdaten}
Die Angabe der Oszilloskop-Messdaten aus \ref{sec:a} ist aufgrund der Menge nicht möglich. Die restlichen Daten finden sich in Tabelle \ref{tab:daten}.

\begin{table}
  %\tiny
  \centering
  \caption{Messdaten zur Frequenzabhängigkeit der Phasenverschiebung und Kondensatorspannung.}
  \label{tab:daten}
  \sisetup{
    round-mode=figures
  }
  \begin{tabular}{
      l@{}
      S[table-format=5.0]
      S[table-format=2.1]
      S[table-format=2.0]
      S[table-format=5.1]
      S[table-format=1.3]
      |
      S[table-format=6.0]
      S[table-format=2.1]
      S[table-format=2.3]
      S[table-format=2.1]
      S[table-format=1.2]
    }
    \toprule
    & {$f / \si{\hertz}$} & {$U/ \si{\volt}$} & {$U_C / \si{\volt}$} & {$a / \si{\milli\second}$} & {$\phi / \si{\radian}$}
    & {$f / \si{\hertz}$} & {$U/ \si{\volt}$} & {$U_C / \si{\volt}$} & {$a / \si{\milli\second}$} & {$\phi / \si{\radian}$}\\
    \midrule
    \input{build/daten.txt}
    %\midrule
    %&\multicolumn{8}{c}{
    %  $\Delta U^\text{rel} = 5\%, \quad \Delta U_0^\text{rel} = 5\%, \quad \Delta a^\text{rel} = 5\%$
    %}\\
    \midrule
    \multicolumn{11}{c}{$\Delta U^\text{rel} = \SI{5}{\%}, \quad \Delta U^\text{rel}_C = \SI{5}{\%}, \quad \Delta a^\text{rel} = \SI{5}{\%}$} \\
    \bottomrule
  \end{tabular}
\end{table}
