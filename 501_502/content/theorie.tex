\section{Ziel}
\label{sec:Ziel}
Es soll die Abhängigkeit der Ablenkung eines Elektronenstrahls von der Stärke eines elektrischen beziehungsweise magnetischen Feldes, sowie die spezifische Ladung des Elektrons und das lokale Erdmagnetfeld bestimmt werden.

\section{Theorie}
\label{sec:theorie}

\subsection{Elektrisches Feld}
Die kinetische Energie eines durch die Spannung $U_B$ in $z$-Richtung beschleunigten Elektrons beträgt
\begin{equation}
  \label{eqn:vz}
  E_\text{kin} = \frac{1}{2} m_e v_z^2 = e \cdot U_B.
\end{equation}
Die Stärke eines in einem (hinreichend langen) Plattenkondensator herrschenden homogenen Elektrische Felds beträgt
\begin{equation}
  E = \frac{U_d}{d}
\end{equation}
mit der anliegenden Spannung $U_d$ und dem Plattenabstand $d$. Mit der elektrischen Kraft
\begin{equation}
  \vec F  = q \vec E
\end{equation}
folgt daraus für die Geschwindigkeit, die ein Elektron im Zeitintervall $\Delta t$ erreicht
\begin{equation}
  v_y = \frac{e}{m_e} \frac{U_d}{d} \Delta t.
\end{equation}
Das Elektron, das wie in Abb.~\ref{fig:ablenkung} mit einer Anfangsgeschwindigkeit $v_z$ durch einen Plattenkondensator der Länge $p$ abgelenkt wird, erfährt demnach eine Ablenkgeschwindigkeit von
\begin{equation}
  v_y = \frac{e U_d p}{m_e d v_z}.
\end{equation}
Unter Näherung der Sinusfunktion für kleine Winkel ergibt sich für den Ablenkwinkel
\begin{equation}
  \theta \approx \frac{v_y}{v_z} = \frac{e U_d p}{m_e d v_z^2}.
\end{equation}
Mit der Entfernung zum Schirm $L$ und \eqref{eqn:vz} ergibt sich auf dem Schirm demnach eine Ablenkung von
\begin{equation}
  \label{eqn:prop}
  D = \frac{p}{2d} L \frac{U_d}{U_B}
\end{equation}
\fig{bilder/2}{Ablenkung eines Elektronenstrahls durch einen Plattenkondensator. \cite{anleitung501502}}{ablenkung}

\subsection{Magnetisches Feld}
Auf eine mit der Geschwindigkeit $\vec v$ bewegte Ladung in einem Magnetfeld $\vec B$ wirkt die Lorentzkraft
\begin{equation}
  \vec {F_L} = q \vec v \times \vec B.
\end{equation}
Wenn das Teilchen (hier Elektron) sich senkrecht zum Magnetfeld bewegt, vereinfacht sich dies zu
\begin{equation}
  F_L = e v_0 B.
\end{equation}
Da die Kraft immer senkrecht zur Bewegungsrichtung steht ist Energie des Teilchens im Gegensatz zum elektrischen Feld konstant, es ändert sich lediglich die Richtung des Geschwindigkeitsvektors. Demnach wirkt die Lorentzkraft als Zentripetalkraft und wird auf eine Kreisbahn gezwungen. Über
\begin{align}
  e m_e B &= \frac{m_e v^2}{r}
\end{align}
und die Tatsache, dass die Geschwindigkeit konstant bleibt folgt dann
\begin{equation}
  r = \frac{m_e v_0}{e B}
\end{equation}
für den Radius der Teilchenbahn, siehe Abb.~\ref{fig:ablenkungB}
Bei einer Anfangsgeschwindigkeit
\begin{equation}
  v_0 = \sqrt{2 U_B \frac{e}{m_e}}
\end{equation}
erhält man über den Satz des Pythagoras die Beziehung
\begin{equation}
  \label{eqn:spez}
  \frac{D}{L^2 + D^2} = \frac{1}{\sqrt{8 U_B}} \sqrt{\frac{e}{m_e}} B
\end{equation}
mit der Länge $L$ und der Ablenkung $D$, siehe wiederum Abb.~\ref{fig:ablenkungB}.
\fig{bilder/6}{Ablenkung eines Elektronenstrahls durch ein homogenes Magnetfeld. \cite{anleitung501502}}{ablenkungB}
