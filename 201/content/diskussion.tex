\section{Diskussion}
\label{sec:Diskussion}

Im Rahmen des Versuches konnten die spezifischen Wärmekapazitäten der beiden Stoffe mit einer Abweichung von $\SI{51.7}{\%}$ (Zinn) bzw. $\SI{44.9}{\%}$ bestimmt werden. Die Literaturwerte liegen bei $c_k^\text{Zinn} = \SI{226}{\joule\per\kilogram\per\kelvin}$, bzw. $c_k^\text{Graphit} = \SI{715}{\joule\per\kilogram\per\kelvin}$ \cite{wärmeatlas}. Demnach sind die Ergebnisse für $C_V$ aussagekräftig genug, um daraus zu schließen, dass die Näherung des Dulong-Petitschen Gesetzes in der richtigen Größenordnung liegt. Allerdings scheinen die Quantenmechanischen Effekte vor allem für das Graphit nicht vernachlässigbar zu sein. Die großen Abweichungen, auch innerhalb der Messung, sind dem Versuchsaufbau zuzuschreiben. Die Temperatur des Probekörpers konnte nicht exakt gemessen werden, da dieser sich im kochenden Wasserbad befand und der Dampf die Messung vermutlich verfälscht hat. Eine Messung außerhalb des Gefäßes war nicht möglich, da die Temperatur unmittelbar stark fiel. Weiterhin ist die Annahme einer konstanten Wärmemenge im Kalorimeter nicht zutreffend, da ein Wärmeaustausch mit der Umgebung stattfinden konnte. Der Körper hat auf dem Weg vom Wasserbad ins Kalorimeter außerdem vermutlich bereits eine signifikante Abkühlung erfahren. 
