\section{Auswertung}
\label{sec:Auswertung}

\subsection{Wärmekapazität der Apparatur}
Die Wärmekapazität des Kalorimeters berechnet sich nach \eqref{eqn:cgmg} zu
\begin{equation}
  c_g m_g = \SI{338.85}{\joule\per\kelvin}.
\end{equation}

\subsection{Wärmekapazitäten der Metalle}
Nach \eqref{eqn:ck} lassen sich die Wärmekapazitäten der Metalle berechnen als
\begin{align}
  c_k^\text{Graphit} &= \SI{1036}{\joule\per\kilogram\per\kelvin} \\
  c_k^\text{Zinn} &= \SI{343 +- 70}{\joule\per\kilogram\per\kelvin},
\end{align}
woraus durch Multiplikation mit der Molmasse und Anwendung von \eqref{eqn:CvCp} die molaren isochoren Wärmekapazitäten
\begin{align}
  C_V^\text{Graphit} &= \SI{12.40}{\joule\per\mol\per\kelvin} \\
  C_V^\text{Zinn} &= \SI{39.0 +- 8}{\joule\per\mol\per\kelvin}
\end{align}
resultieren. Zur Evaluation des Dulong-Petitschen Gesetzes werden diese nun noch mit $3R$ verglichen:
\begin{align}
  \frac{C_V^\text{Graphit}}{3R} &= 0.4971 \\
  \frac{C_V^\text{Zinn}}{3R} &= 1.563
\end{align}

\subsection{Messdaten}
Messung von $c_gm_g$:
\begin{align}
  m_\text{Glas} &= \SI{187.63}{g}\\
  m_\text{Glas+Wasser} &= \SI{729.12}{g}\\
  U_\text{vorher} &= \SI{0.84}{mV}\\
  U_\text{nachher}&= \SI{2.25}{mV}\\
  U_\text{Wasser} &= \SI{4.13}{mV}
\end{align}

\begin{table}
  \caption{Messdaten.}
  \centering
  \label{tab:par}
  \begin{tabular}{l@{} c S[table-format=1.2] S[table-format=1.2] S[table-format=1.3] S[table-format=3.2] S[table-format=3.2] S[table-format=3.2]}
    \toprule
    & Material & $U_k/\si{mV}$ & $U_w/\si{mV}$ & $U_m/\si{mV}$ & $m_k/\si{g}$ & $m_\text{Deckel}/\si{g}$ & $m_\text{Wasser+Glas}/\si{g}$ \\
    \midrule
    \input{build/daten.tex}
    \bottomrule
  \end{tabular}
\end{table}
