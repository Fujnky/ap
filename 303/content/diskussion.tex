\section{Diskussion}
\label{sec:Diskussion}

Der Graph der Messwerte ist so wie \ref{eqn:ganzwichtig} eine Cosinus-Funktion. Sie sind in der Phase identisch. Die Amplitude der Messwerte liegt über der des anderen Graphen, was auf Schwankungen bei der Anzeige der Spannung zurückgeführt werden kann.
Folglich kann \ref{eqn:ganzwichtig} als korrekt angenommen werden.

Bei der Photodetektorschaltung nimmt die Spannung in Abhänigkeit vom Abstand potentiell ab, da die sich ausbreitende Welle kugelförmig ist.
