\section{Diskussion}
\label{sec:Diskussion}

Der Graph der Messwerte entspricht (\ref{eqn:ganzwichtig}) eine Cosinus-Funktion. Sie sind in der Phase identisch. Die Amplitude der Messwerte liegt über der des anderen Graphen, was auf Schwankungen bei der Anzeige der Spannung zurückgeführt werden kann.
Folglich kann (\ref{eqn:ganzwichtig}) als korrekt angenommen werden.

Wie erwartet nimmt bei der Photodetektorschaltung die Spannung in Abhängigkeit vom Abstand exponentiell ab, da die sich ausbreitende Welle kugelförmig ist. Die exponentielle Abnahme lässt sich an der fallenden Geraden im Doppellogarithmischen Plot erkennen.

Der Lock-In-Verstärker kann selbst sehr verrauschte Signale, welche auf dem Oszilloskop nicht mehr erkennbar sind, messen.
