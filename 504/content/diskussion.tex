\section{Diskussion}
\label{sec:Diskussion}

Bei den fünf aufgenommenen Kennlinien konnte die Sättigungsspannung lediglich von dreien ermittelt werden, da die Beschleunigungsspannung zu gering m, um überall eine Sättigung zu erzeugen. Bei den niedrigen Strömen sind die verschiedenen Abschnitte der Kennlinie gut zu erkennen.
Der ermittelte Exponent des Langmuir-Schottkyschen Raumlandgungsgesetzes liegt \SI{3.3}{\percent} unter dem Exponenten $\sfrac{3}{2}$ der Formel.
Die durch den Stromverlauf im Anlaufgebiet ermittelte Temperatur liegt in einem plausiblen Bereich für einen Glühfaden, weiterhin in derselben Größenordnung wie die durch Abschätzung ermittelten Temperaturen aus der anderen Messreihe. Die berechnete Austrittsarbeit liegt allerdings \SI{23}{\percent} unter dem Literaturwert von $\phi_\text{lit} = \SI{4.54}{\electronvolt}$\cite{spektrum_austrittsarbeit}. Dies kann durch eine ungenaue Schätzung der Kathodentemperatur begründet werden.

\DeclarePairedDelimiter\bra{\langle}{\rvert}
\DeclarePairedDelimiter\ket{\lvert}{\rangle}
\DeclarePairedDelimiterX\braket[2]{\langle}{\rangle}{#1 \delimsize\vert #2}

\begin{equation}
\ket*{\dot{\widetilde{\widehat{\overline{\underline{\underline{K^{\mu \nu \dagger \top *'}_{\kappa'}}}}}}}}
\end{equation}
