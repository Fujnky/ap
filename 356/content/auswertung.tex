\section{Auswertung}
\label{sec:Auswertung}

\subsection{Durchlasskurven}
Die aufgenommenen Durchlasskurven für die verschiedenen Ketten sind in Abb.~\ref{fig:durchl1} und \ref{fig:durchl2} aufgetragen (Die Plots wurden gescannt und danach wurde mittels eines Programms die Kurve in Daten umgewandelt). Die ermittelten Grenzfrequenzen ergeben sich zu
\begin{align}
  \omega_g^{LC} &= \SI{49.85}{\kilo\hertz}
  \shortintertext{und}
  \omega_g^{LC_1C_2} &= \SI{35.42}{\kilo\hertz}
\end{align}
\fig{build/a1.pdf}{Durchlasskurve der $LC$-Kette}{durchl1}
\fig{build/a2.pdf}{Durchlasskurve der $LC_1C_2$-Kette}{durchl2}

\subsection{Dispersionskurven}
Zunächst wird die LC-Kette betrachtet.Für die Phasenverschiebung pro Glied gilt:
\begin{equation}
  \theta = \frac{\phi}{n}
\end{equation}
mit $\theta$ als Phasenverschiebung pro Kettenglied, $\phi$ als Phasenverschiebung über die ganze Kette und $n$ als Anzahl der Kettenglieder.
Außerdem gilt $\omega = 2\pi f$.

\begin{table}
  \centering
  \caption{Messwerte zur Dispersionsrelation der LC-Kette.}
  \label{tab:dispersion1}
  \begin{tabular}{c c c c}
    \toprule
    $\phi$ & $\theta$ & $f / \si{\hertz}$ & $\omega / \si{\hertz}$ \\
    \midrule
$\pi$ & 0.196 & 5009 & 31472.48 \\
$2\pi$ & 0.393 & 10036 & 63058.05 \\
$3\pi$ & 0.589 & 14941 & 93877.07 \\
$4\pi$ & 0.785 & 19678 & 123640.52 \\
$5\pi$ & 0.982 & 24471 & 153755.83 \\
$6\pi$ & 1.178 & 28517 & 179177.50 \\
$7\pi$ & 1.374 & 32447 & 203870.51 \\
$8\pi$ & 1.571 & 36158 & 227187.41 \\
$9\pi$ & 1.768 & 39595 & 248782.72 \\
$10\pi$ & 1.963 & 42676 & 268141.22 \\
$11\pi$ & 2.160 & 45015 & 282837.59 \\
$12\pi$ & 2.356 & 47255 & 296911.92 \\
\bottomrule
\end{tabular}
\end{table}

\begin{figure}
  \centering
  \includegraphics[width=\textwidth]{build/dispersion1.pdf}
\caption{Dispersionsrelation einer LC-Kette.}
  \label{fig:dispersion-lc}
\end{figure}

Zur Darstellung der Dispersionskurve wird $\omega$ gegem $\theta$ aufgetragen. Neben den Messwerten wird eine Theoriekurve dargestellt, welche sich aus
\begin{equation}
  \omega = \sqrt{\frac{2}{LC}(1-\cos(\theta))}
\end{equation}
ergibt.

 Nun soll die Dispersionskurve für die $LC_{1}C{2}$-Kette bestimmt werden.
 Für die Dispersionsrelation gilt:
 \begin{equation}
  \omega = \sqrt{\frac{1}{L}\left(\frac{1}{C_{1}}+\frac{1}{C_{2}}\right)\pm\frac{1}{L}\sqrt{\left(\frac{1}{C_{1}+\frac{1}{C_2}}\right)^2 + \frac{4\sin^2(\theta)}{C_{1}C_{2}}}}.
  \end{equation}
  Nach dieser Formel ergibt sich die Theoriekurve der Dispersionsrelation einer $LC_{1}C{2}$-Kette.

\begin{table}
  \centering
  \caption{Messwerte zur Dispersionsrelation der $LC_{1}C_{2}$-Kette.}
  \label{tab:dispersion1}
  \begin{tabular}{c c c c}
    \toprule
    $\phi$ & $\theta$ & $f / \si{\hertz}$ & $\omega / \si{\hertz}$ \\
    \midrule
    $\pi$ & 0.196 & 5743 & 36084.333 \\
    $2\pi$ & 0.393 & 11141 & 70000.968 \\
    $3\pi$ & 0.589 & 16960 & 106562.823 \\
    $4\pi$ & 0.785 & 22187 & 139405.032 \\
    $5\pi$ & 0.982 & 27087 & 170192.64 \\
    $6\pi$ & 1.178 & 31400 & 197292.019 \\
    $7\pi$ & 1.374 & 34790 & 218592.017 \\
    $8\pi$ & 1.571 & 38857 & 244145.731 \\
    \bottomrule
    \end{tabular}
  \end{table}

  \begin{figure}
    \centering
    \includegraphics[width=\textwidth]{build/dispersion2.pdf}
  \caption{Dispersionsrelation einer $LC_{1}C_{2}$-Kette.}
    \label{fig:dispersion-lc1c2}
  \end{figure}

  \subsection{Phasengeschwindigkeit}
  Aus den Messwerten lässt sich die Phasengeschwindigkeit aus
  \begin{equation}
    v_\mathrm{ph} = \frac{\omega}{\theta}
  \end{equation}
  berechnen. Auch hier ist $\theta$ die Phasenverschiebung pro Kettenglied.
  Die zum Vergleich dienende Theoriekruve ergibt sich aus \ref{eqn:v theorie}.

  \begin{table}
    \centering
    \caption{Messwerte zur Phasengeschwindigkeut in der LC-Kette.}
    \label{tab:dispersion1}
    \begin{tabular}{c c c c c}
      \toprule
      $\phi$ & $\theta$ & $f / \si{\hertz}$ & $\omega / \si{\hertz}$ & $v_\mathrm{ph}$ \\
      \midrule
$\pi$ & 0.196 & 5055 & 31761.502 & 162048.478 \\
$2\pi$ & 0.393 &  9922 &  62781.588 & 159749.587 \\
$3\pi$ & 0.589 & 14938 & 93858.222 & 159351.587 \\
$4\pi$ & 0.785 & 19262 & 123313.795 & 157087.637 \\
$5\pi$ & 0.982 & 24187 & 151971.40 & 154757.03 \\
$6\pi$ & 1.178 & 28510 & 179133.61 & 152065.885 \\
$7\pi$ & 1.374 & 32438 & 203813.97 & 148336.219 \\
$8\pi$ &  1.571 & 36279 & 227947.68 & 145097.186 \\
$9\pi$ & 1.767 & 45057 & 283101.48 & 160125.272 \\
$10\pi$ &  1.963 & 47168 & 296365.285 & 150975.693 \\
$11\pi$ & 2.160 & 48720 & 306116.788 & 141720.735 \\
$12\pi$ & 2.356 & 49847 & 313197.938 & 132936.306 \\
$13\pi$ & 2.553 & 50754 & 318896.787 & 124910.61 \\
\bottomrule
\end{tabular}
\end{table}

Da sich die Oberschwingungen zunächst im Abstand von ungefähr 4000 \si{\Hz} befinden, lässt sich vermuten, dass sich eine weitere bei ungefähr 40000 \si{\Hz} befindet, die wegen der ungenauen Frequenzregelung nicht gefunden wurde.
Die Ursache für die deutlich über der Theoriekurve liegenden Messwerte, lässt sich auf die zuvor erwähnte fehlende Oberschwingung zurückführen.

\begin{figure}
  \centering
  \includegraphics[width=\textwidth]{build/v-ph.pdf}
\caption{Phasengeschwindigkeit in einer LC-Kette.}
  \label{fig:v-ph}
\end{figure}

\subsection{Nachweis stehender Wellen}

Für die erste Eigenschwingungen ergeben sich die Spannungen in \ref{tab:eigenschwingung1}. Der Wellenwiderstand ist zunächst nicht angeschlossen.
\begin{table}
  \centering
  \caption{Messwerte zur 1.Eigenschwingung der offenen LC-Kette.}
  \label{tab:eigenschwingung1}
  \begin{tabular}{c c}
    \toprule
    $n$ & $U / \si{\volt}$ \\
\midrule
1 & 1.90 \\
2 & 1.85 \\
3 & 1.75 \\
4 & 1.60 \\
5 & 1.35 \\
6 & 1.05 \\
7 & 0.71 \\
8 & 0.38 \\
9 & 0.01 \\
10 & 0.34 \\
11 & 0.66 \\
12 & 0.94 \\
13 & 1.15 \\
14 & 1.25 \\
15 & 1.40 \\
16 & 1.42 \\
17 & 1.43 \\
\bottomrule
\end{tabular}
\end{table}

\begin{figure}
  \centering
  \includegraphics[width=\textwidth]{build/eigenschwingung1.pdf}
\caption{1. Eigenschwingung der LC-Kette.}
  \label{fig:eigenschwingung1}
\end{figure}

Für die zweite Eigenschwingung ergeben sich die Werte in \ref{tab:eigenschwingung2}

\begin{table}
  \centering
  \caption{Messwerte zur 2.Eigenschwingung der offenen LC-Kette.}
  \label{tab:eigenschwingung2}
  \begin{tabular}{c c}
    \toprule
    $n$ & $U / \si{\volt}$ \\
\midrule
1 & 1.80 \\
2 & 1.60 \\
3 & 1.20 \\
4 & 0.64 \\
5 & 0.26 \\
6 & 0.66 \\
7 & 1.25 \\
8 & 1.65 \\
9 & 1.80 \\
10 & 1.65 \\
11 & 1.25 \\
12 & 0.66 \\
13 & 0.006 \\
14 & 0.64 \\
15 & 1.15 \\
16 & 1.60 \\
17 & 1.75 \\
\bottomrule
\end{tabular}
\end{table}

\begin{figure}
  \centering
  \includegraphics[width=\textwidth]{build/eigenschwingung2.pdf}
\caption{2. Eigenschwingung der LC-Kette.}
  \label{fig:eigenschwingung2}
\end{figure}

Die mit angeschlossenem Wellenwiderstand aufgenommenen Messwerte sind in \ref{tab:wellenwiderstand} dargestellt.
\begin{table}
  \centering
  \caption{Messwerte zur 2.Eigenschwingung der offenen LC-Kette.}
  \label{tab:eigenschwingung2}
  \begin{tabular}{c c}
    \toprule
    $n$ & $U / \si{\volt}$\\
    \midrule
1 & 0.59 \\
2 & 0.38 \\
3 & 0.15 \\
4 & 0.49 \\
5 & 0.60 \\
6 & 0.40 \\
7 & 0.14 \\
8 & 0.46 \\
9 & 0.59 \\
10 & 0.42 \\
11 & 0.13 \\
12 & 0.44 \\
13 & 0.60 \\
14 & 0.43 \\
15 & 0.125 \\
16 & 0.43 \\
17 & 0.60 \\
\bottomrule
\end{tabular}
\end{table}

\begin{figure}
  \centering
  \includegraphics[width=\textwidth]{build/wellenwiderstand.pdf}
\caption{Messwerte der Spannungen entlang der Kette für angeschlossene Wellenwiderstände.}
  \label{fig:wellenwiderstand}
\end{figure}

Durch Anschließen des Wellenwiderstandes sollen Reflexionen am Ende und die daraus resultierenden stehenden Wellen verhindert werden. Dies ist nicht vollständig gewährleistet. Es finden Teilreflexionen statt, die mit der hinlaufenden Welle interferieren und somit zu Schwankungen in der Amplitude führen. Die hohen Schwankungen können durch die große Frequenz erklärt werden.
