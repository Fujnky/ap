\section{Diskussion}
\label{sec:Diskussion}
Die Schwankungen den Durchlasskurven werden von dem frequenzabhängigen Wellenwiderstand verursacht. Außerdem konnte dieser nicht exakt eingestellt werden. Die Messwerte zur Dispersion der LC-Kette zeigen nur geringe Abweichungen zur Theoriekurve. Die der $LC_{1}C_{2}$-Kette zeigen auf dem unteren Ast ebenfalls geringe Abweichungen. Auf dem oberen optischen Ast liegen keine Messwerte, da das Oszilloskop bei hohen Freuquenzen nicht mehr so eingestellt werden konnte, sodass die Lissajous-Figur als Gerade auf dem Bildschrim sichtbar war. Somit wurde dieser Frequenzbereich nicht ausgemessen. Bei der Ermittlung der Phasengeschwindigkeit ist darauf hinzuweisen, dass eine Eigenschwingung auf Grund grober Frequenzeinstellung fehlt.
Bei der Darstellung der Eigenschwingungen fällt auf, dass die zweite Halbwelle jeweils eine geringere Amplitude besitzt, was auf ohmsche Verluste zurückgeführt werden kann.
