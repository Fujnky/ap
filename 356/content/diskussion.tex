\section{Diskussion}
\label{sec:Diskussion}

Die ermittelten Grenzfrequenzen liegen bei   $f_g^{LC} = \SI{49,85}{\kilo\hertz},
f_{g,1}^{LC_1C_2} = \SI{35,42}{\kilo\hertz},
f_{g,2}^{LC_1C_2} = \SI{55,04}{\kilo\hertz},
f_{g,3}^{LC_1C_2} = \SI{64,84}{\kilo\hertz}$. Sie stimmen im Fall der $LC_1C_2$-Kette mit den berechneten Grenzfrequenzen überein, bei der $LC$-Kette ist die Abweichung erheblich.
Die Schwankungen der Durchlasskurven werden von dem frequenzabhängigen Wellenwiderstand verursacht. Außerdem konnte dieser nicht exakt eingestellt werden. Die Messwerte zur Dispersion der LC-Kette zeigen nur geringe Abweichungen zur Theoriekurve. Die der $LC_{1}C_{2}$-Kette zeigen auf dem unteren Ast ebenfalls geringe Abweichungen. Auf dem oberen optischen Ast liegen keine Messwerte, da das Oszilloskop bei hohen Freuquenzen nicht mehr so eingestellt werden konnte, sodass die Lissajous-Figur als Gerade auf dem Bildschrim sichtbar war. Somit wurde dieser Frequenzbereich nicht ausgemessen. Bei der Ermittlung der Phasengeschwindigkeit ist darauf hinzuweisen, dass eine Eigenschwingung auf Grund grober Frequenzeinstellung fehlt. Die aufgenommenen Werte liegen bis \SI{227}{\kilo\Hz} auf der Theoriekurve. Erst nach der fehlenden Oberschwingung zeigen sich geringe Abweichungen.
 In  den Abbildungen \ref{fig:eigenschwingung1} und \ref{fig:eigenschwingung2} sind deutlich stehende Wellen zu erkennen. Bei der Darstellung der Eigenschwingungen fällt auf, dass die zweite Halbwelle jeweils eine geringere Amplitude besitzt, was auf ohmsche Verluste zurückgeführt werden kann.
