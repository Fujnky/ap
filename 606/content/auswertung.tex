\section{Auswertung}
\label{sec:Auswertung}

\subsection{Selektivverstärker}
Um die Güte des Selektivverstärker zu bestimmen, wird die Ausgangsspannung $U_\mathrm{A}$ gegen die Frequenz $f$ aufgetragen. Die Messwerte hierzu befinden sich in Tabelle \ref{tab:verstaerker}.

\begin{table}
  \caption{Messwerte zur Ausgangsspannung $U_\mathrm{A}$ und zur Frequenz $f$}
  \centering
  \label{tab:verstaerker}
  \begin{tabular}{c c }
    \toprule
   $U_\mathrm{A}/\si{\milli\volt}$&  $f$ /\si{\kilo\Hz}\\
    \midrule
    30 & 30 \\
    31 & 40\\
    32 & 50\\
    33 & 70\\
    34 & 140\\
    35 & 690\\
    36 & 220\\
    37 & 100\\
    38 & 60\\
    39 & 45\\
    40 & 35\\
    33.5 & 100\\
    34.5 & 250\\
    35.5 & 580\\
    36.5 & 140\\
    37.5 & 80\\
    33.75  &120\\
    34.25 & 180\\
    34.75 & 370\\
    35.25 & 1183\\
    35.75 & 340\\
    36.25 & 180\\
    35.1 & 925\\
    35.2 & 1200\\
    35.3 & 1100\\
    35.4 & 830\\
    34.9 & 500\\
    \bottomrule
    \end{tabular}
\end{table}

In Abbildung \ref{fig:verst} ist die Filterkurve des Selektivverstärkers graphisch dargestellt. Die Güte berechnet sich wie folgt:
\begin{equation}
  Q=\frac{f_0}{f_+ - f_-}.
\end{equation}
Dabei ist $f_0$ die Durchlassfrequenz und $f_+$ und $f_-$ liegen jeweils an der Stelle, wo die Ausgangsspannnung auf das $\frac{1}{\sqrt{2}}$-fache des Maximums erreicht hat.
Mit
\begin{align}
  f_0 &= 35,2 \si{\kilo\Hz}\\
  f_+ &= 35,4 \si{\kilo\Hz}\\
  f_- &= 35,05 \si{\kilo\Hz}\\
\end{align}
ergibt sich eine Güte von $Q=100,571$.

\begin{figure}
  \centering
  \includegraphics{build/a.pdf}
\caption{Filterkurve des Selektivverstärkers.}
  \label{fig:verst}
\end{figure}

\subsection{Bestimmung der Suszeptibilitäten}
Um zu berücksichtigen, dass die aus staubförmigem Material bestehenden Proben nur eingeschränkt dicht gestopft werden können, wird der Querschnitt $Q_\mathrm{real}$ berechnet.
\begin{equation}
  Q_\mathrm{real}=\frac{m}{L \rho}
\end{equation}
$L$ ist die Länge der Probe, $m$ die Masse und $\rho$ die Dichte.
Die entsprechenden Werte sind in \ref{tab:daten} zu finden. Die Messwerte zu den verschiedenen Widerständen und Spannungen sind in \ref{tab:messwerte} dargestellt. $U_1$ ist dabei die abgeglichene Brückenspannung ohne Probe, $U_2$ Brückenspannung mit Probe, $U_3$ die abgeglichene Brückenspannung mit Probe, $R_1$ der Widerstand ohne und $R_2$ mit Probe. Die Speisespannung beträgt $U_\mathrm{Sp}=0,98 \si{\volt}$ und der Querschnitt der Spule $F=86,6 \cdot 10^{-6} \si{\meter}^2$.

\begin{table}
  \caption{Masse, Länge, Dichte und Querschnitt der Proben.}
  \centering
  \label{tab:daten}
  \begin{tabular}{c c c c c}
    \toprule
    Probe &  $m /10^{-3}\si{\kilo\gram}$ & $L$/\si{\meter} & $\rho/10^3 \si{\kilo\gram}/\si{\meter}^3$ & $Q/10^{-6}\si{\meter}^2$\\
    \midrule
    $Gd_2 O_3$ & 14,08 & 0,147 & 7,4 & 12,94 \\
    $Nd_2 O_3$ & 9 & 0,149 & 7,24 & 8,34 \\
    $Dy_2 O_3$ & 18,5 & 0,149 & 7,8 & 1,59 \\
    \bottomrule
    \end{tabular}
  \end{table}

  \begin{table}
    \caption{Messwerte zur Bestimmung der Suszeptibilitäten.}
    \centering
    \label{tab:messwerte}
    \begin{tabular}{c c c c c c}
      \toprule
      Probe & $U_1$/\si{\milli\volt} & $U_2$/\si{\milli\volt} & $U_3$/\si{\milli\volt} & $R_1$/ 5\cdot \si{\milli\ohm} &$R_2$/ 5\cdot\si{\milli\ohm} \\
      \midrule
    $Gd_2 O_3$ & 4,65 & 38,0 & 5,80 & 719 & 559 \\
     - & 4,70 & 38,5 & 5,30 & 720,5 & 557 \\
     - & 4,60 & 39,0 & 5,50 & 720,0 & 554 \\
    $Nd_2 O_3$ & 4,65 & 8,9 & 4,70 & 720,0 & 688 \\
    - & 4,65 & 8,5 & 4,65 & 717,5 & 717 \\
    - & 4,65 & 8,0 & 4,75 & 714,0 & 698 \\
    $Dy_2 O_3$ & 4,65 & 91,5 & 9,10 & 719,5 & 338 \\
    - & 4,70 & 89,0 & 9,30 & 718,0 & 337 \\
    - & 4,65 & 89,0 & 9,50 & 719,5 & 339 \\
    \bottomrule
    \end{tabular}
  \end{table}

  $\chi_1$ berechnet sich nach Gleichung \ref{eqn:chi2} und $\chi_2$ nach Gleichung \ref{eqn:chi3}. Mit den gemittelten Werten für die Spannungen und Widerstände aus Tabelle \ref{tab:messwerte} ergeben sich für die drei Proben die Werte in \ref{tab:suszep}.

  \begin{table}
    \caption{Berechnete Suszeptibilitäten.}
    \centering
    \label{tab:suszep}
    \begin{tabular}{c c c}
      \toprule
      Probe & $\chi_1$ & $\chi_2$ \\
      \midrule
  $Gd_2 O_3$ & 0,00925 \pm 0,00008  & 0,01105 \pm 0,00010 \\
  $Nd_2 O_3$ & 0,00153 \pm 0,0010 & 0,0027 \pm 0,0013 \\
  $Dy_2 O_3$ & 0,03608 \pm 0,00035 & 0,04003 \pm 0,00008 \\
  \bottomrule
  \end{tabular}
\end{table}

Zusätzlich sollen die Suszeptibilitäten mit Gleichung \ref{eqn:chi1} berechnet werden. Dazu müssen zunächst die Quantenzahlen und daraus der Landé-Faktor $g_\mathrm{J}$  nach Gleichung \ref{eqn:lande} bestimmt werden. Die Ergebnisse sind in \ref{tab:quantenzahlen} aufgeführt. Für $N$ aus Gleichung \ref{eqn:chi1} gilt:
\begin{equation}
  N=\frac{2 \rho N_\mathrm{A}}{M}
\end{equation}
mit $M$ als molare Masse und $N_\mathrm{A} = 6,022 \cdot 10^{23} \frac{1}{\si{\mol}}$ als Avogadro-Konstante \cite{codata}.

\begin{table}
  \caption{Quantenzahlen der Proben.}
  \centering
  \label{tab:quantenzahlen}
  \begin{tabular}{c c c c c c}
    \toprule
    - & $L$ & $S$ & $J$ & $g_\mathrm{j}$ & $N/10^{28}$ \\
    \midrule
  $Gd_2 O_3$ & 0 & 3,5 & 3,5 & 2 & 2,459 \\
  $Nd_2 O_3$ & 6 & 1,5 & 4,5 & 0,72 & 2,591 \\
  $Dy_2 O_3$ & 5 & 2,5 & 7,5 & 4/3 & 2,519 \\
  \bottomrule
  \end{tabular}
\end{table}

Für die Suszeptibilitäten ergeben sich damit die Werte in Tabelle \ref{tab:chi-theo}.

\begin{table}
  \caption{Theoriewerte für die Suszeptibilitäten.}
  \centering
  \label{tab:chi-theo}
  \begin{tabular}{c c}
    \toprule
    Probe & $\chi_\mathrm{theoretisch}$ \\
    \midrule
  $Gd_2 O_3$ & 0,014 \\
  $Nd_2 O_3$ & 0,003 \\
  $Dy_2 O_3$ & 0,025 \\
  \bottomrule
  \end{tabular}
\end{table}

Die Abweichungen der berechneten Werte $\chi_1$ und $\chi_2$ von den Theoriewerten $\chi_\mathrm{theoretisch}$ sind in Tabelle \ref{tab:abweichungen} aufgeführt.

\begin{table}
  \caption{Abweichungen der berechneten Suszeptibilitäten $\chi_1$ und $\chi_2$ vpn den Theoriewerten.}
  \centering
  \label{tab:abweichungen}
  \begin{tabular}{c c c}
    \toprule
    Probe & Abweichung $\chi_1$ & Abweichung $\chi_2$ \\
    \midrule
  $Gd_2 O_3$ & (51,4 \pm 1,3)\% & (26,7 \pm 1,2) \% \\
  $Nd_2 O_3$ & (97,0 \pm 13)\% & (10,0 \pm 50)\% \\
  $Dy_2 O_3$ & (44,3 \pm 1,4)\% & (60,1 \pm 0,03)\% \\
  \bottomrule
  \end{tabular}
\end{table}
