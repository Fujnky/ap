\subsection{Fehlerrechnung}

\subsubsection{Mittelwert und Standardabweichung}
Der Mittelwert mehrerer Messwerte wird berechnet durch
\begin{equation}
\langle v\rangle = \frac{1}{N} \sum_{i=1}^N v_i,
\end{equation}
dabei ist die Standardabweichung
\begin{equation}
s_i = \sqrt{\frac{1}{N - 1} \sum_{j=1}^N \left(v_j - \langle v\rangle\right){^2}},
\end{equation}
wobei $v_j$ ($j = 1, ..., N$) die Messwerte sind.
Der Standardfehler ist über
\begin{equation}
\sigma_i = \frac{s_i}{\sqrt{N}} = \sqrt{\frac{\sum_{j=1}^N \left(v_j - \langle v_i\rangle\right){^2}}{N \left(N - 1 \right)}}.
\end{equation}
definiert.


\subsubsection{Gaußfehler}
Bei einer fehlerbehafteten Funktion $f$ mit $k$ als fehlerbehafteter Größe und $\sigma_k$ als Ungenauigkeit, gilt
\begin{equation}
\Delta x_k = \frac{\mathrm{d}f}{\mathrm{d}k}\sigma_k.
\end{equation}

Der relative Gaußfehler berechnet sich nach
\begin{equation}
\Delta x_\text{k, rel} = 1 \pm \frac{\Delta x_k}{|x|}\cdot 100\%.
\end{equation}

Der absolute Gaußfehler ergibt sich aus
\begin{equation}
\Delta x_i = \sqrt{\left(\frac{\mathrm{d}f}{\mathrm{d}k_{1}}\cdot \sigma_{k_{1}}\right)^2 + \left(\frac{\mathrm{d}f}{\mathrm{d}k_{2}}\cdot \sigma_{k_{2}}\right)^2 + ...}.
\end{equation}

\subsubsection{Lineare Regression}
\label{sec:linregress}
Bei einer linearen Regression über den Messdaten ${x_i, y_i}$ wird für die Steigung
\begin{align}
  m &= \frac{\langle x y \rangle - \langle x \rangle \langle y \rangle}{\langle x^2 \rangle - \langle x \rangle ^2}
  \intertext{und für den $y$-Achsenabschnitt}
  b &= \langle y \rangle - m \cdot \langle x\rangle
\end{align}
angenommen. Für die Standardabweichung gelten
\begin{align}
  s_m &= \sqrt{\frac{1}{N-2} \sum^N_{i=1} (y_i - b - mx_i)^2}
  \shortintertext{und}
  s_b &= s_m \cdot \sqrt{\frac{1}{N (\langle x^2 \rangle - \langle x \rangle ^2)}}.
\end{align}

%\subsubsection{Lineare Regression}
%\begin{equation}
%\sigma {^2} = \sum_{k=1}^{N} \left(y_k - \left(\frac{\overline{xy} - \overline{x}\cdot\overline{y}}{\overline{x^2} - (\overline{x}){^2}}x_k + \left(\overline{y} - %\overline{B}\overline{x}\overline{B}\right)\right)\right){^2}
%\end{equation}
