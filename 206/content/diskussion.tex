\section{Diskussion}
\label{sec:Diskussion}

Im Vergleich zur idealen Wärmepumpe (siehe Tabelle 1) ist die Güteziffer der im Versuch verwendeten Wärmepumpe deutlich schlechter. Der erste Grund dafür besteht darin, dass idealisierend angenommen wird, dass der Prozess reversibel verläuft. Dies gilt nicht im realistischen Fall. Eine weitere idealisierende Annahme in der Rechnung ist, dass die Kompression adiabatisch erfolgt. Auch dies ist realistisch betrachtet nicht erfüllt.

Außerdem treten Energieverluste durch Reibung und durch Wärmeleitung auf. Durch nicht ausreichende Isolierung (nicht passende Deckel) kann ein Wärmeaustausch mit der umgebenden Luft stattfinden, was zu einem Anstieg der Temperatur $T_{2}$ im kälteren sowie zu einem Sinken der Temperatur $T_{1}$ im wärmeren Reservoir führt.

Desweiteren sind die Skalen der Manometer grob skaliert, sodass das genaue Ablesen des Drucks, der Zeit und der Verdampfungswärme nur eingeschränkt möglich ist.
