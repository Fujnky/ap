\section{Durchführung}
\label{sec:Durchführung}
\subsection{Prinzip}
In einer Wärmepumpe wird Energie von einem Reservoir in ein anderes befödert. Dies geschieht dadurch, dass ein Medium (\enquote{Kältemittel}) beim Durchlaufen des zu kühlenden Reservoirs verdampft und ihm dabei Wärme entzieht. Als nächstes Glied im Kreislauf folgt ein Kompressor (K in Abb. \ref{fig:wärmepumpe}), der den Druck des Mediums erhöht, bevor es im zu erhitzenden Reservoir (1) wieder kondensiert und dabei Energie an des Reservoir abgibt. Im Reiniger (R) wird die Flüssigkeit dann von Gasresten befreit, woraufhin sie das Drosselventil (D) passiert, das einen Druckunterschied zwischen $p_a$ und $p_b$ im Kreislauf erzeugt. Damit ist der Kreislauf geschlossen. Das Ventil ist durch eine Steuervorrichtung (S) gesteuert, die verhindert, dass flüssiges Medium in den Kompressor gelangen kann und ihn somit zerstört.

\begin{figure}
  \centering
  \def\svgwidth{\columnwidth}
  \input{content/Schema_Waermepumpe.pdf_tex}
  \caption{Schematischer Aufbau einer Wärmepumpe.}
  \label{fig:wärmepumpe}
\end{figure}

\subsection{Versuchsaufbau}
Der konkrete Versuchsaufbau in unserem Experiment unterscheidet sich nur gering vom allgemeinen Schema. Hinzu kommen Rührmotoren, die eine gleichmäßige Umwälzung des Wassers in den Reservoirs und damit eine gleichmäßige Temperaturverteilung gewährleisten (sowie ein vorzeitiges Einfrieren verhindern). Zusätzlich zu den Drücken $p_a$ und $p_b$ werden noch die Temperaturen der Reservoirs $T_{1}$ und $T_{2}$, sowie die elektrische Leistung des Kompressors $P$ zu unterschiedlichen Zeitpunkten $t$ gemessen.

\begin{figure}
  \centering
  \input{content/Aufbau.pdf_tex}
  \caption{Versuchsaufbau.}
  \label{fig:aufbau}
\end{figure}
