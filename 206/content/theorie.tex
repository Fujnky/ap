\section{Theorie}
\label{sec:Theorie}

Im folgenden Experiment wird die Wärmepumpe, welche die Wärmeenergie zwischen zwei Wärmereservoirs transportiert, untersucht. Um den Transport vom kälteren zum wärmeren Reservoir zu ermöglichen, muss Energie aufgewendet werden, was durch die Wärmepumpe gewährleistet wird.

\subsection{Güteziffer}
Aus dem ersten und zweiten Hauptsatz der Thermodynamik folgt die Formel für die Güteziffer einer idealen Wärmepumpe, welche das Verhältnis der Wärmemenge und der aufzuwendenden Arbeit beschreibt.

\begin{equation}
\label{eqn:güte_ideal}
ν_{\mathrm{ideal}} = \frac{Q_{1}}{A} = \frac{T_{1}}{T_{1}-T_{2}}
\end{equation}

mit $Q_{1}$ als abgegebene Wärmemenge an das wärmere Reservoir, $Q_{2}$ als dem wärmeren Reservoir entzogene Wärmemenge, $T_{1}$ als Temperatur des wärmeren und $T_{2}$ als Temperatur des kälteren Reservoirs.

Die Güteziffer der idealisierten Wärmepumpe ergibt sich aus der Annahme, dass der Prozess reversibel ist. Da im realistischen Fall der Prozess nicht vollständig reversibel ist, gilt nun:

\begin{equation}
ν_{\mathrm{real}} < \frac{T_{1}}{T_{1} - T_{2}}
\end{equation}

Je kleiner die Temperaturdifferenz der Reservoirs desto kleiner ist die aufzuwendende Arbeit.
Im folgenden soll die reale Güteziffer berechnet werden. Diese errechnet sich aus $\frac{\Delta T_{1}}{\Delta t}$, woraus sich die Güteziffer bestimmen lässt.
Aufgrund von Ausgleichsrechnung durch eine Fitfunktion wird im weiteren Verlauf mit Differentialquotienten anstelle von Differenzenquotienten gerechnet.

\begin{equation}
\label{eqn:güteziffer1}
\frac{\mathrm{d}Q_{1}}{\mathrm{d}t} = (m_{1} c_w + m_k c_k)\frac {\mathrm{d}T_{1}}{\mathrm{d}t}
\end{equation}

Dabei ist $m_{1} c_w$ die Wärmekapazität des Wassers im Reservoir 1 und $m_k c_k$ entspricht der Wärmekapazität der Kupferschlange und des Eimers. Somit gilt für die Güteziffer

\begin{equation}
\label{eqn:güteziffer2}
ν = \frac{\mathrm{d}Q_{1}}{\mathrm{d}t N}
\end{equation}

mit N als gemittelte Leistungsaufnahme des Kompressors.

\subsection{Massendurchsatz}
Auch der Massendurchsatz lässt sich mit Hilfe des Differentialquotienten berechnen. Für die entzogene Wärmemenge des zweiten Reservoirs gilt

\begin{equation}
\label{eqn:durchsatz1}
\frac{\mathrm{d}Q_{2}}{\mathrm{d}t} = (m_2 c_w + m_kc_k)\frac{\mathrm{d}T_{2}}{\mathrm{d}t}
\end{equation}

mit $m_{2} c_w$ als Wärmekapazität des Wassers im Reservoir 2.

Der Wärmeentzug erfolgt durch Verdampfung und deshalb gilt außerdem

\begin{equation}
\label{eqn:durchsatz2}
\frac{\mathrm{d}Q_{2}}{\mathrm{d}t} = L\frac{\mathrm{d}m}{\mathrm{d}t}
\end{equation}

Bei bekannter Verdampfungswärme L kann der Massendurchsatz berechnet werden.

\subsection{Bestimmung der mechanischen Kompressorleistung}
Um das Volumen von $V_a$ auf $V_b$ zu reduzieren, leistet der Kompressor die Arbeit $A_m$

\begin{equation}
A_m = - \int_{V_a}^{V_b} p \, \mathrm{d}V
\end{equation}

Nimmt man an, dass es sich um eine adiabatische Kompression handelt, gilt die Poissonsche Gleichung.

\begin{equation}
p_a V_a^{\kappa} = p_b V_b^{\kappa} = pV^{\kappa}
\end{equation}

Damit folgt für die mechanische Kompressorleistung


\begin{equation}
\label{eqn:mechleistung}
N_\text{mech} = \frac{\mathrm{d}A_m}{\mathrm{d}t} = \frac{1}{\kappa -1}\biggl(p_b \sqrt[\kappa]{\frac{p_a}{p_b}}-p_a\biggr)\frac{\mathrm{d}V_a}{\mathrm{d}t} = \frac{1}{\kappa-1}\biggl(p_b \sqrt[\kappa]{\frac{p_a}{p_b}}-p_a\biggr)\frac{1}{\rho} \frac{\mathrm{d}m} {\mathrm{d}t}
\end{equation}

mit $\rho$ als Dichte des gasförmigen Transportmediums (beim Druck $p_a$), welches man aus der idealen Gasgleichung bestimmen kann.

\section{Aufgabe}
\label{sec:Aufgabe}
Ziel ist es, den Transport der Wärmeenergien zwischen den Reservoiren zu untersuchen. Dabei werden unter anderem die Güteziffer sowie der Massendurchsatz näher betrachtet.
%\cite{sample}
