\section{Diskussion}
\label{sec:Diskussion}

Die Abweichung des Schubmoduls um $(21,8 \pm 0,009)\%$ liegt innerhalb des Toleranzbereiches, da es mehrere Messunsicherheiten gibt, die nicht weiter berücksichtigt werden. Zum Einen wird das Schwingen des Pendels durch Bewegung des Tisches, wie zum Beispiel leichtes Anstoßen beeinflusst. Außerdem kann der Magnet nicht exakt in Nord-Süd Richtung ausgerichtet werden, sodass das Erdmagnetfeld die Messung trotz dieser Einrichtung beeinflusst. Weiterhin ist zu beachten, dass die in der Theorie angenomme Kleinwinkelnäherung nicht immer angehalten wurde, was zu weiteren Fehlern führt.
Bei der Bestimmung der Horizontalkomponente des Erdmagnetfelds ist lediglich ein kleiner Fehler von $(11,6 \pm 1,7)\%$ verglichen mit dem Literaturwert$B_\mathrm{horizontal} = 20 \cdot 10^{-6} \si{\tesla}$  \cite{dornbader} festzustellen.
Alle Messwerte liegen auf der Ausgleichsgeraden, mit welcher das magnetische Moment bestimmt werden soll. Der berechnete Wert für das magnetische Moment liegt in einer realistischen Größenordnung.
