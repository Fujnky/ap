\section{Zielsetzung}
\label{sec:Ziel}
Mithilfe des Kugelfall-Viskosimeters nach Höppler soll die Temperaturabhängigkeit
der dynamischen Viskosität destillierten Wassers bestimmt werden. Außerdem
wird durch Ermittlung der Reynoldszahl eine Aussage über die Art der Strömung
getroffen.

\section{Theorie}
\label{sec:Theorie}
Die dynamische Viskosität $\eta$ einer Flüssigkeit ist eine temperaturabängige
Materialkonstante, welche die Größe der auf einen sich durch die Flüssigkeit bewegenden
Körper wirkenden Kräfte angibt. Fällt etwa ein Körper durch eine Flüssigkeit,
so wirken auf ihn die Gewichtskraft $\vec{F_G}$ und dieser entgegengesetzt
die Auftriebskraft $\vec{F_A}$ und die Reibungskraft $\vec{F_R}$.
\newline
\newline
Bei dem Kugelfall-Viskosimeter nach Höppler wird die Viskosität dadurch bestimmt,
dass eine Kugel durch die Flüssigkeit fällt. Nach einer gewissen Zeit gleichen sich
die auf die Kugel wirkenden Kräfte aus und die Fallgeschwindigkeit bleibt konstant.
Die Viskosität ergibt sich dann aus der Formel
\begin{equation}
  \label{eqn:eta}
  \eta = K (\rho_\symup{K}-\rho_\symup{Fl}) \cdot t \, .
\end{equation}
$K$ ist dabei eine Apparaturkonstane, welche Fallhöhe und Kugelgeometrie beinhält.
$\rho_\symup{K}$ bzw. $\rho_\symup{Fl}$ sind die Dichten der Kugel bzw. der Flüssigkeit,
$t$ ist die Fallzeit.
\newline
Die Temperaturabhängigkeit der Viskosität ist durch die Andradesche Gleichung
\begin{equation}
  \label{eqn:etaT}
  \eta(T) = A \; \exp \left(\frac{B}{T} \right)
\end{equation}
gegeben, wobei $A$ und $B$ Konstanten sind.
\newline
\newline
Zur Bestimmung der wirkenden Kräfte ist die Art der Strömung wichtig. Es wird eine
laminare, also nicht turbulente Strömung gefordert. Eine laminare Strömung zeichnet
sich dadurch aus, dass sich keine Wirbel bilden. Um zu ermitteln um welche Strömung es sich handelt,
wird die Reynoldszahl $Re$
\begin{equation}
  \label{eqn:reynolds}
  Re = \frac{\rho v d}{\eta}
\end{equation}
bestimmt, wobei $\rho$ die Dichte der Flüssikeit, $v$ die Strömungsgeschwindigkeit,
$d$ die charakteristische Länge des Körpers in Strömungsrichtung und $\eta$ die dynamische Viskosität
darstellen.
\newline
\newline
Handelt es sich um eine laminare Strömung, ergibt sich für die Reibung der
Flüssigkeitsschischten die Stokessche Reibung
\begin{equation}
  \label{eqn:stokes}
  F_R = 6 \pi \eta \, v \,r \; ,
\end{equation}
wobei $r$ der Radius der fallenden Kugel ist.

\section{Fehlerrechnung}
Der Mittelwert wird über die Gleichung
\begin{equation}
  \label{eqn:MW}
   \overline{x_i} = \frac{1}{n} \sum_{i=1}^n  x_i
\end{equation}
bestimmt.
\newline
Der Fehler dieses Mittelwerts ergibt sich durch die Gleichung
\begin{equation}
  \label{eqn:MF}
  \increment{x}=\frac{\sigma}{\sqrt{n}}\, ,
\end{equation}
wobei $\sigma$ die Standardabweichung
\begin{equation}
  \sigma = \sqrt{\frac{1}{n-1}\sum_{i=1}^n (x_i-\overline{x_i})^2}
\end{equation}
darstellt.
\newline
Wird eine Größe aus fehlerbehafteten Werten errechnet, ergibt sich der Fehler durch
die Gaußsche Fehlerfortpflanzung
\begin{equation}
  \label{eqn:gauß}
  \sigma_f = \sqrt{\left(\frac{\partial f}{\partial x_1}\right)^2\cdot \increment x_1^2
  + \left(\frac{\partial f}{\partial x_2}\right)^2\cdot \increment x_2^2 + ...
  + \left(\frac{\partial f}{\partial x_n}\right)^2\cdot \increment x_n^2 } \, .
\end{equation}
