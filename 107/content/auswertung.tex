\section{Auswertung}
\label{sec:Auswertung}
\subsection{Messung bei Raumtemperatur}
Die Masse und Durchmesser der Kugeln sind
\begin{align*}
  m_\text{kl} &= \SI{4.43}{\gram} \\
  m_\text{gr} &= \SI{4.95}{\gram} \\
  d_\text{kl} &= \SI{15.311}{\milli\meter} \\
  d_\text{gr} &= \SI{15.492}{\milli\meter},
\end{align*}
daraus ergeben sich die Dichten
\begin{align*}
  \rho_\text{kl} &= \SI{2354}{\kilogram\per\cubic\meter} \\
  \rho_\text{gr} &= \SI{2543}{\kilogram\per\cubic\meter}.
\end{align*}
Die Dichte von Wasser bei der Messtemperatur $\SI{21.5}{\celsius}$ liegt nach \cite{wärmeatlas} bei
\begin{align*}
  \rho_\text{w} = \SI{998.21}{\kilogram\per\cubic\meter}.
\end{align*}
Mit diesen Größen, der gegebenen Gerätekonstante
\begin{align*}
  K_\text{kl} = \SI{0.07640}{\milli\pascal\cubic\centi\meter\per\gram}
\end{align*}
und der gemittelten Fallzeit
\begin{align*}
  \overline{t_\text{kl}} = \SI{12.4+-0.1}{s}
\end{align*}
ergibt sich nach \eqref{eqn:eta} eine dynamische Viskosität von
\begin{align}
  \eta_\text{kl} = \SI{1.28+-0.01}{\pascal\second}.
\end{align}
Anhand dieser Größe lässt sich zusammen mit der gemittelten Fallzeit der Zweiten Messreihe
\begin{align*}
  \overline{t_\text{gr}} = \SI{80.9+-0.6}{s}
\end{align*}
die Gerätekonstante der Apparatur mit der großen Kugel zu
\begin{align}
  K_\text{gr} = \SI{1.03+-0.01e-8}{\pascal\cubic\meter\per\kilogram}
\end{align}
bestimmen.

\begin{table}
  \caption{Messdaten der ersten Messreihe.}
  \centering
  \label{tab:d1}
  \begin{tabular}{l@{} S S}
    \toprule
    & {$t_\text{kl}/\si{s}$} & $t_\text{gr}/\si{s}$ \\
    \midrule
    & 12.46 & 82.03 \\
& 12.47 & 81.84 \\
& 12.31 & 80.21 \\
& 12.47 & 80.34 \\
& 12.28 & 81.13 \\
& 12.41 & 81.13 \\
& 12.35 & 80.29 \\
& 12.13 & 80.29 \\
& 12.35 & 81.13 \\
& 12.50 & 81.06 \\

    \bottomrule
  \end{tabular}
\end{table}

\subsection{Messung der Temperaturabhängigkeit}

Um $\eta$ bei verschiedenen Temperaturen zu ermitteln, muss jeweils die Dichte des Wassers bekannt sein. Dazu werden die in Tabelle~\ref{tab:dichte} gegebenen Literaturwerte mittels nichtlinearer Ausgleichsrechnung an ein quadratisches Polynom interpoliert, siehe Abb~\ref{fig:dichte}.

\begin{table}
  \caption{Literaturwerte zur Dichte von Wasser im Messbereich. Entnommen aus \cite{wärmeatlas}.}
  \centering
  \label{tab:dichte}
  \begin{tabular}{l@{} S S}
    \toprule
    & {$T/\si{K}$} & $\rho_\text{Wasser}/\si{\kilogram\per\cubic\meter}$ \\
    \midrule
    \input{build/dichte.tex}
    \bottomrule
  \end{tabular}
\end{table}

\fig{build/dichte}{Literaturwerte zur Dichte von Wasser, interpoliert mit einer quadratischen Funktion.}{dichte}
\FloatBarrier
Nun kann wiederum anhand \eqref{eqn:eta} und der zuvor bestimmten Apparaturkonstante $K_\text{gr}$ die Viskosität für jede Temperatur bestimmt werden. Die Ergebnisse finden sich in Tabelle~\ref{tab:erg} und Abb.~\ref{fig:erg}.

\begin{table}
  \caption{Ergebnisse der Berechnungen im temperaturabhängigen Teil der Messung.}
  \centering
  \label{tab:erg}
  \begin{tabular}{l@{} S S S
      S[table-format=2.1, round-precision=3, round-mode=figures] @{${}\pm{}$} S[table-format=1.2, round-precision=1, round-mode=figures]
      S[table-format=3.1, round-precision=4, round-mode=figures]
      S[table-format=1.3, round-precision=3, round-mode=figures] @{${}\pm{}$} S[table-format=1.3, round-precision=1, round-mode=figures]
      S[table-format=2.1, round-precision=3, round-mode=figures] @{${}\pm{}$} S[table-format=1.1, round-precision=1, round-mode=figures]}
    \toprule
     &{$T/\si{K}$} & {$t_1/\si{s}$} & {$t_2/\si{s}$} & \multicolumn{2}{c}{$t/\si{s}$} & {$\rho_\text{Wasser}/\si{\kilogram\per\cubic\meter}$} & \multicolumn{2}{c}{$\eta/\si{\milli\pascal\second}$} & \multicolumn{2}{c}{$Re$} \\
    \midrule
    \begin{table}
        \caption{Ergebnisse der Auswertung, erster Teil.}
        \centering
        \label{tab:erg}
        \sisetup{
          table-align-uncertainty=false,
          table-number-alignment = center
        }
        \begin{tabular}{l@{}S[table-format=1.2e2, round-precision=3, round-mode=figures] S[table-format=2.3, round-precision=3, round-mode=figures] @{${}\pm{}$} S[table-format=1.3, round-precision=1, round-mode=figures] S[table-format=2.2, round-precision=3, round-mode=figures] @{${}\pm{}$} S[table-format=1.2, round-precision=1, round-mode=figures] S[table-format=4.2, round-precision=3, round-mode=figures] @{${}\pm{}$} S[table-format=2.1, round-precision=1, round-mode=figures] } \toprule & {$m/\si{\ohm}$}& \multicolumn{2}{c}{$R_H/\si{(10^{-11} \cubic\meter\per\coulomb)}$}& \multicolumn{2}{c}{$d/\si{µm}$}& \multicolumn{2}{c}{$n/\si{\per\cubic\nano\meter}$}\\\midrule \multicolumn{8}{c}{Kupfer} \rule{0pt}{3ex}\\& 1.34e-06 & 0.27770171824967693208 & 0.00355053136101397262 & 2.19000301849452760905 & 0.02800009465871976938 & 2247.55869903247048569028 & 28.73596784683861216081 \\
 \multicolumn{8}{c}{Zink} \rule{0pt}{3ex}\\& 7.73e-05 & 90.33695095675079755893 & 1.56739920622705519726 & 12.37149511054998995974 & 0.21465271310077210787 & 6.90914300270263392889 & 0.11987769283169560919 \\
 \bottomrule \end{tabular} \end{table}

    \bottomrule
  \end{tabular}
\end{table}

\fig{build/viskosität}{Berechnete Viskosität von Wasser, aufgetragen über die Temperatur; zusätzlich mit nichtlinearem Fit.}{erg}

Um nun die materialspezifischen Größen $A$ und $B$ der Andradeschen Gleichung \eqref{eqn:etaT} zu bestimmen, wird $\ln(\eta)$ über $T^{-1}$ aufgetragen. Durch
\begin{align}
  \ln(\eta(T^{-1})) &= \ln\left(A \cdot \exp \left(\frac{B}{T^{-1}}\right)\right) \\
  \ln(\eta) &= \ln(A) + BT
\end{align}
ist demnach $\ln(A)$ der $\eta$-Achsenabschnitt und $B$ die Steigung der Geraden. Durch die lineare Regression ergeben sich die folgenden Werte:
\begin{align*}
  A &= \SI{3.78+-0.29e-06}{\pascal\second}, \\
  B &= \SI{1706+-24}{\kelvin}
\end{align*}
eine zusätzlich durchgeführte nichtlineare Ausgleichsrechnung ergab dieselben Werte. Zusätzlich wird noch für jede Temperatur die Reynoldszahl nach \eqref{eqn:reynolds} bestimmt. Im vorliegenden Versuchsaufbau entspricht $d$ dem Rohrdurchmesser, welcher durch den Durchmesser der fallenden Kugel hinreichend genau angenährt werden kann. Die Ergebnisse finden sich ebenfalls in Tabelle~\ref{tab:erg}.

\fig{build/viskosität_linear}{Logarithmierte Viskositätswerte über reziproke Temperatur aufgetragen, sowie lineare Regression zur Bestimmung einer Geradengleichung.}{ergl}
