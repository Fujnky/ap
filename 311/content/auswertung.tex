\section{Auswertung}
\label{sec:Auswertung}

\section{Fehlerrechnung}
\label{subsec:fehlerrechnung}

\subsubsection{Mittelwert}
\begin{equation}
\overline{v} = \frac{1}{N} \sum_{i=1}^N v_i
\end{equation}

\subsubsection{Standardabweichung}
\begin{equation}
s_i = \sqrt{\frac{1}{N - 1} \sum_{j=1}^N \left(v_j - \overline{v}\right){^2}}
\end{equation}

wobei $v_j$ ($j = 1, ..., N$) die Messwerte sind.

\subsubsection{Streuung der Mittelwerte}
\begin{equation}
\sigma_i = \frac{s_i}{\sqrt{N}} = \sqrt{\frac{\sum_{j=1}^N \left(v_j - \overline{v_i}\right){^2}}{N \left(N - 1 \right)}}
\end{equation}

\subsubsection{Gaußfehler}
Bei einer fehlerbehafteten Funktion $f$ mit $k$ als fehlerbehafteter Größe und $\sigma_k$ als Ungenauigkeit, gilt:

\begin{equation}
\Delta x_k = \frac{\mathrm{d}f}{\mathrm{d}k}\sigma_k
\end{equation}.

 Der relative Gaußfehler berechnet sich nach:
\begin{equation}
\Delta x_\text{k, rel} = 1 \pm \frac{\Delta x_k}{|x|}\cdot 100\%
\end{equation}.

Der absolute Gaußfehler ergibt sich aus:
\begin{equation}
\Delta x_i = \sqrt{\left(\frac{\mathrm{d}f}{\mathrm{d}k_{1}}\cdot \sigma_{k_{1}}\right)^2 + \left(\frac{\mathrm{d}f}{\mathrm{d}k_{2}}\cdot \sigma_{k_{2}}\right)^2 + ...}
\end{equation}.


\subsection{Leitfähigkeitsparameter}
Die gemessenen Hallspannungen werden über den Strom durch die Probe aufgetragen und es wird eine lineare Regression der Daten durchgeführt und damit die Steigung der Geraden $m$ bestimmt. Das bedeutet, dass eine Ausgleichsrechnung an die Funktion
\begin{equation}
  B(I) = m I + n
\end{equation}
durchgeführt wird, siehe \ref{sec:linregress}. Die zugehörigen Plots finden sich in Abb.~\ref{fig:b} und Abb.~\ref{fig:b1}. Da die Dicken der Proben unbekannt sind (oder bei Kupfer als unbekannt angenommen werden soll), werden sie über \begin{equation}
  d = \frac{\rho L}{R b}
\end{equation}
mit dem spezifischen Widerstand $\rho$ (aus \cite{kkfkfkfkfkfkfkfkfkfkfkfkkfkfkfkfkkfkfkfkfkoekfoefkwpokwpoekfpowkefpowkefpowkefpowkpokwefpowkfpoewkfpowekfpoewkfpowkefpokefpowkefpowekfpoewkfpowekf}) bestimmt. Damit kann dann über \begin{equation}
  R_H = \frac{m}{bd} = \frac{1}{ne}
\end{equation}
die Hall-Konstante bestimmt werden.
\fig{build/b2.pdf}{Hallspannung in Abhängigkeit von der Stromstärke für Kupfer.}{b1}
\fig{build/b.pdf}{Hallspannung in Abhängigkeit von der Stromstärke für Zink.}{b}

Aus $R_H$ kann wiederum auf triviale Art und Weise die Ladungsträgerdichte bestimmt werden. Aus
\begin{equation}
  z = n  V_m / N_A
\end{equation}
kann mit dem molaren Volumen, das jeweils \cite{chemie.de} entnommen ist, die Zahl der Ladungsträger pro Atom bestimmt werden. Für weitere Berechnugen ist der Ohmsche Widerstand der Probe nötig, der über
\begin{equation}
  R = \frac{U}{I}
\end{equation}
aus der Widerstandsmessreihe (gemittelt) bestimmt wird. Für die mittlere Flugzeit folgt aus \eqref{eqn:S}
\begin{equation}
  \bar\tau = \left(\frac{e^2 R Q n}{2 m_e L}\right)^{-1}.
\end{equation}
Die mittlere Driftgeschwindigkeit
\begin{equation}
  \bar v_d = -\frac{j}{ne}
\end{equation}
wird für
\begin{equation}
  j = \SI{1}{\ampere\per\milli\meter\squared}
\end{equation}
berechnet.
Die Totalgeschwindigkeit
\begin{equation}
  v \approx \sqrt{\frac{2 E_F}{m_e}}
\end{equation}
und die mittlere freie Weglänge
\begin{equation}
  \bar l \approx \bar \tau \sqrt{\frac{2 E_F}{m_e}}
\end{equation}
werden aus den zuvor berechneten Größen bestimmt. Alle Ergebnisse der angegebenen Berechnungen finden sich in den Tabellen~\ref{tab:erg},~\ref{tab:erg2}~und~\ref{tab:erg3}.
\begin{table}
        \caption{Ergebnisse der Auswertung, erster Teil.}
        \centering
        \label{tab:erg}
        \sisetup{
          table-align-uncertainty=false,
          table-number-alignment = center
        }
        \begin{tabular}{l@{}S[table-format=1.2e2, round-precision=3, round-mode=figures] S[table-format=2.3, round-precision=3, round-mode=figures] @{${}\pm{}$} S[table-format=1.3, round-precision=1, round-mode=figures] S[table-format=2.2, round-precision=3, round-mode=figures] @{${}\pm{}$} S[table-format=1.2, round-precision=1, round-mode=figures] S[table-format=4.2, round-precision=3, round-mode=figures] @{${}\pm{}$} S[table-format=2.1, round-precision=1, round-mode=figures] } \toprule & {$m/\si{\ohm}$}& \multicolumn{2}{c}{$R_H/\si{(10^{-11} \cubic\meter\per\coulomb)}$}& \multicolumn{2}{c}{$d/\si{µm}$}& \multicolumn{2}{c}{$n/\si{\per\cubic\nano\meter}$}\\\midrule \multicolumn{8}{c}{Kupfer} \rule{0pt}{3ex}\\& 1.34e-06 & 0.27770171824967693208 & 0.00355053136101397262 & 2.19000301849452760905 & 0.02800009465871976938 & 2247.55869903247048569028 & 28.73596784683861216081 \\
 \multicolumn{8}{c}{Zink} \rule{0pt}{3ex}\\& 7.73e-05 & 90.33695095675079755893 & 1.56739920622705519726 & 12.37149511054998995974 & 0.21465271310077210787 & 6.90914300270263392889 & 0.11987769283169560919 \\
 \bottomrule \end{tabular} \end{table}

\begin{table}
        \caption{Ergebnisse der Auswertung, zweiter Teil.}
        \centering
        \label{tab:erg2}
        \begin{tabular}{l@{}S[table-format=3.2, round-precision=3, round-mode=figures] @{${}\pm{}$} S[table-format=2.2, round-precision=1, round-mode=figures] S[table-format=3.2, round-precision=3, round-mode=figures] @{${}\pm{}$} S[table-format=1.2, round-precision=1, round-mode=figures] S[table-format=1.3, round-precision=3, round-mode=figures] @{${}\pm{}$} S[table-format=1.3, round-precision=1, round-mode=figures] S[table-format=3.2, round-precision=3, round-mode=figures] @{${}\pm{}$} S[table-format=1.2, round-precision=1, round-mode=figures] } \toprule & \multicolumn{2}{c}{$\bar v_d/\si{(\milli\meter\per\second)}$}& \multicolumn{2}{c}{$\mu/\si{(\centi\meter\squared\per\volt\per\second)}$}& \multicolumn{2}{c}{$v/\si{(\mega\meter\per\second)}$}& \multicolumn{2}{c}{$\bar l/\si{nm}$}\\\midrule \multicolumn{9}{c}{Kupfer} \rule{0pt}{3ex}\\& 2.77701718249676998695 & 0.03550531361013972015 & 3.22721346019380517234 & 0.04126125927965104689 & 4.69135012028878684021 & 0.01999361741147750657 & 8.60803723605021176013 & 0.07337154497040515588 \\
 \multicolumn{9}{c}{Zink} \rule{0pt}{3ex}\\& 903.36950956750797558925 & 15.67399206227055685758 & 301.12316985583595396747 & 5.22466402075685110162 & 0.68213273619422631899 & 0.00394513466131994270 & 116.78623681151960056468 & 1.35087324317788604056 \\
 \bottomrule \end{tabular} \end{table}
\begin{table}
        \caption{Ergebnisse der Auswertung, dritter Teil.}
        \centering
        \label{tab:erg3}
        \begin{tabular}{l@{}S[table-format=2.3, round-precision=3, round-mode=figures] @{${}\pm{}$} S[table-format=1.3, round-precision=1, round-mode=figures] S[table-format=1.2, round-precision=3, round-mode=figures] @{${}\pm{}$} S[table-format=1.1, round-precision=1, round-mode=figures] S[table-format=3.2, round-precision=3, round-mode=figures] @{${}\pm{}$} S[table-format=1.2, round-precision=1, round-mode=figures] } \toprule & \multicolumn{2}{c}{$z$}& \multicolumn{2}{c}{$R/\si{\milli\ohm}$}& \multicolumn{2}{c}{$\bar\tau/\si{fs}$}\\\midrule \multicolumn{7}{c}{Kupfer} \rule{0pt}{3ex}\\& 26.53565024395918925393 & 0.33926926693110143196 & 8.80144905610693584208 & 0.11253016759506800915 & 1.83487418660629098000 & 0.02345962561601189905 \\
 \multicolumn{7}{c}{Zink} \rule{0pt}{3ex}\\& 0.10509177949763809368 & 0.00182340415545403402 & 8.20745274771785560119 & 0.14240413015575315026 & 171.20749469245029672493 & 2.97055068207408679015 \\
 \bottomrule \end{tabular} \end{table}
\FloatBarrier
\subsection{Hysteresekurve}
Die magnetische Flussdichte des Elektromagneten in Abhängigkeit vom Spulenstrom findet sich in Abb.~\ref{fig:hyst}.

\fig{build/hysterese.pdf}{Hysteresekurve des verwendeten Elektromagneten. Zur Verdeutlichung des Kurvenverlaufs wurden die Messwerte mit einem Polygonzug interpoliert.}{hyst}
\FloatBarrier
\subsection{Messdaten}
\input{build/datena.tex}
\begin{table}
        \caption{Messdaten, Hallspannungsmessreihe.}
        \centering
        \label{datena}
        \begin{tabular}{
          l@{}
          S[round-mode=off, table-format=1.1]
          S[round-mode=off, table-format=3.0]
          S[round-mode=off, table-format=1.0]
          |
          S[round-mode=off, table-format=2.1]
          S[round-mode=off, table-format=3.0]
          S[round-mode=off, table-format=2.0]
        }
        \toprule
        & {$I/\si{A}$}& {$U_\text{Zn}/\si{µV}$}& {$U_\text{Cu}/\si{µV}$}& {$I/\si{A}$}& {$U_\text{Zn}/\si{µV}$}& {$U_\text{Cu}/\si{µV}$}\\
        \midrule
        &  0.0 & 0   & 0 & 5.5  & 428 & 7  \\
        &  0.5 & 41  & 1 & 6.0  & 468 & 8  \\
        &  1.0 & 82  & 2 & 6.5  & 503 & 9  \\
        &  1.5 & 115 & 2 & 7.0  & 545 & 10 \\
        &  2.0 & 155 & 3 & 7.5  & 579 & 10 \\
        &  2.5 & 195 & 4 & 8.0  & 620 & 11 \\
        &  3.0 & 235 & 4 & 8.5  & {}  & 12 \\
        &  3.5 & 270 & 5 & 9.0  & {}  & 12 \\
        &  4.0 & 308 & 6 & 9.5  & {}  & 13 \\
        &  4.5 & 348 & 6 & 10.0 & {}  & 14 \\
        &  5.0 & 386 & 7 &   {} & {}  & {} \\
 \bottomrule \end{tabular} \end{table}

\begin{table}
        \caption{Messdaten, Hysteresekurve.}
        \centering
        \label{de}
        \begin{tabular}{l@{}cc|cc|cc} \toprule & {$I/\si{A}$}& {$B/\si{mT}$}& {$I/\si{A}$}& {$B/\si{mT}$}& {$I/\si{A}$}& {$B/\si{mT}$}\\\midrule& 0,0 & 27,05 & 3,5 & 792,5 & 3,0 & 730,1 \\
& 0,5 & 125,2 & 4,0 & 888,8 & 2,5 & 623,7 \\
& 1,0 & 242,6 & 4,5 & 978,0 & 2,0 & 508,0 \\
& 1,5 & 364,7 & 5,0 & 1059 & 1,5 & 394,5 \\
& 2,0 & 480,2 & 4,5 & 1000 & 1,0 & 274,3 \\
& 2,5 & 590,5 & 4,0 & 921,8 & 0,5 & 150,4 \\
& 3,0 & 692,7 & 3,5 & 829,6 & 0,0 & 27,67 \\
 \bottomrule \end{tabular} \end{table}

