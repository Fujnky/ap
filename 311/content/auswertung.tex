\section{Auswertung}
\label{sec:Auswertung}
\subsection{Leitfähigkeitsparameter}
Die gemessenen Hallspannungen werden über den Strom durch die Probe aufgetragen und es wird eine lineare Regression der Daten durchgeführt und damit die Steigung der Geraden $m$ bestimmt. Die zugehörigen Plots finden sich in Abb.~\ref{fig:b} und Abb.~\ref{fig:b1}. Da für Kupfer die Dicke der Probe bekannt ist, kann dort die Hallkonstante
\begin{equation}
  R_\text{H} = \frac{1}{ne}
\end{equation}
gemäß \eqref{eqn:hall} bestimmt werden. Es wird ein Magnetfeld von $B = \SI{1.059}{T}$ angenommen (siehe nächster Abschnitt). Für Zink ist dies nicht möglich, es wird stattdessen ein Literaturwert für die Hallkonstante aus \cite{zinkhall} benutzt, um die Dicke zu bestimmen.

\fig{build/b2.pdf}{Hallspannung in Abhängigkeit von der Stromstärke für Kupfer.}{b1}
\fig{build/b.pdf}{Hallspannung in Abhängigkeit von der Stromstärke für Zink.}{b}

Aus $R_H$ kann wiederum auf triviale Art und Weise die Ladungsträgerdichte bestimmt werden. Aus
\begin{equation}
  z = n  V_m / N_A
\end{equation}
kann mit dem molaren Volumen, das jeweils \cite{chemie.de} entnommen ist, die Zahl der Ladungsträger pro Atom bestimmt werden. Für weitere Berechnugen ist der Ohmsche Widerstand der Probe nötig, der über
\begin{equation}
  R = \frac{U}{I}
\end{equation}
aus der Widerstandsmessreihe (gemittelt) bestimmt wird. Für die mittlere Flugzeit folgt aus \eqref{eqn:S}
\begin{equation}
  \bar\tau = \left(\frac{e^2 R Q n}{2 m_e L}\right)^{-1}.
\end{equation}
Die mittlere Driftgeschwindigkeit
\begin{equation}
  \bar v_d = -\frac{j}{ne}
\end{equation}
wird für
\begin{equation}
  j = \SI{1}{\ampere\per\milli\meter\squared}
\end{equation}
berechnet.
Die Totalgeschwindigkeit
\begin{equation}
  v \approx \sqrt{\frac{2 E_F}{m_e}}
\end{equation}
und die mittlere freie Weglänge
\begin{equation}
  \bar l \approx \bar \tau \sqrt{\frac{2 E_F}{m_e}}
\end{equation}
werden aus den zuvor berechneten Größen bestimmt. Alle Ergebnisse der angegebenen Berechnungen finden sich in den Tabellen~\ref{tab:erg}~und~\ref{tab:erg2}.
\begin{table}
        \caption{Ergebnisse der Auswertung, erster Teil.}
        \centering
        \label{tab:erg}
        \sisetup{
          table-align-uncertainty=false,
          table-number-alignment = center
        }
        \begin{tabular}{l@{}S[table-format=1.2e2, round-precision=3, round-mode=figures] S[table-format=2.3, round-precision=3, round-mode=figures] @{${}\pm{}$} S[table-format=1.3, round-precision=1, round-mode=figures] S[table-format=2.2, round-precision=3, round-mode=figures] @{${}\pm{}$} S[table-format=1.2, round-precision=1, round-mode=figures] S[table-format=4.2, round-precision=3, round-mode=figures] @{${}\pm{}$} S[table-format=2.1, round-precision=1, round-mode=figures] } \toprule & {$m/\si{\ohm}$}& \multicolumn{2}{c}{$R_H/\si{(10^{-11} \cubic\meter\per\coulomb)}$}& \multicolumn{2}{c}{$d/\si{µm}$}& \multicolumn{2}{c}{$n/\si{\per\cubic\nano\meter}$}\\\midrule \multicolumn{8}{c}{Kupfer} \rule{0pt}{3ex}\\& 1.34e-06 & 0.27770171824967693208 & 0.00355053136101397262 & 2.19000301849452760905 & 0.02800009465871976938 & 2247.55869903247048569028 & 28.73596784683861216081 \\
 \multicolumn{8}{c}{Zink} \rule{0pt}{3ex}\\& 7.73e-05 & 90.33695095675079755893 & 1.56739920622705519726 & 12.37149511054998995974 & 0.21465271310077210787 & 6.90914300270263392889 & 0.11987769283169560919 \\
 \bottomrule \end{tabular} \end{table}

\begin{table}
        \caption{Ergebnisse der Auswertung, zweiter Teil.}
        \centering
        \label{tab:erg2}
        \begin{tabular}{l@{}S[table-format=3.2, round-precision=3, round-mode=figures] @{${}\pm{}$} S[table-format=2.2, round-precision=1, round-mode=figures] S[table-format=3.2, round-precision=3, round-mode=figures] @{${}\pm{}$} S[table-format=1.2, round-precision=1, round-mode=figures] S[table-format=1.3, round-precision=3, round-mode=figures] @{${}\pm{}$} S[table-format=1.3, round-precision=1, round-mode=figures] S[table-format=3.2, round-precision=3, round-mode=figures] @{${}\pm{}$} S[table-format=1.2, round-precision=1, round-mode=figures] } \toprule & \multicolumn{2}{c}{$\bar v_d/\si{(\milli\meter\per\second)}$}& \multicolumn{2}{c}{$\mu/\si{(\centi\meter\squared\per\volt\per\second)}$}& \multicolumn{2}{c}{$v/\si{(\mega\meter\per\second)}$}& \multicolumn{2}{c}{$\bar l/\si{nm}$}\\\midrule \multicolumn{9}{c}{Kupfer} \rule{0pt}{3ex}\\& 2.77701718249676998695 & 0.03550531361013972015 & 3.22721346019380517234 & 0.04126125927965104689 & 4.69135012028878684021 & 0.01999361741147750657 & 8.60803723605021176013 & 0.07337154497040515588 \\
 \multicolumn{9}{c}{Zink} \rule{0pt}{3ex}\\& 903.36950956750797558925 & 15.67399206227055685758 & 301.12316985583595396747 & 5.22466402075685110162 & 0.68213273619422631899 & 0.00394513466131994270 & 116.78623681151960056468 & 1.35087324317788604056 \\
 \bottomrule \end{tabular} \end{table}

\subsection{Hysteresekurve}
Die magnetische Flussdichte des Elektromagneten in Abhängigkeit vom Spulenstrom findet sich in Abb.~\ref{fig:hyst}.

\fig{build/hysterese.pdf}{Hysteresekurve des verwendeten Elektromagneten. Zur Verdeutlichung des Kurvenverlaufs wurden die Messwerte mit einem Polygonzug interpoliert.}{hyst}
