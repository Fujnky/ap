\section{Diskussion}
\label{sec:Diskussion}

Die Messdaten lassen sich gut durch eine lineare Regression fitten. Bei Kupfer ist die Linearität schlechter als bei Zink, da der Messbereich deutlich geringer war und die Grenze der Messgenauigkeit des Geräts erreicht wurde. Die Leitfähigkeitsparameter liegen in der korrekten Größenordnung, allerdings liegen keine Literaturdaten zum Vergleich vor. Die Durchführung des Versuchs war problematisch, da viele Proben nicht ordnungsgemäßes Verhalten zeigten. Die Hysteresekurve des Elektromagneten zeigt ein plausibles Verhalten, da die Magnetfeldstärke bei steigendem Strom höher war als bei sinkendem Strom. Dies ist durch den Ferromagnetismus des Elektromagneten zu begründen.
