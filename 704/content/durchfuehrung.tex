\section {Aufbau und Durchführung}
\label{sec:durchführung}

\fig{bilder/aufbau.pdf}{Schmematischer Aufbau zur Messung von $\gamma$- und $\beta$-Strahlung \cite{anleitung704}.}{aufbau}

Zur Messung der Absoprtion von $\gamma$- und $\beta$-Strahlung wird der in Abbildung \ref{fig:aufbau} dargstellte Aufbau in Verbindung mit einem Elektronischen Zählwerk verwendet.
Zu Beginn ist eine Nullmessung in einem Zeitraum von \SI{900}{\second} erforderlich, um auftretende Hintergrundstrahlung in den nachfolgenden Messungen berücksichtigen zu können. Anschließend wird der $\gamma$-Strahler aus $\ce{^{137}Cs}$ in der Apparatur platziert und die Aktivitäten für jeweils 15 Platten mit verschiedener Dicke aus Eisen und Kupfer gemessen.
Dieser Vorgang wird für den $\beta^-$-Strahler, welcher aus $\ce{^{99}Tc}$ besteht, für 11 Platten aus Aluminium verschiedner Dicke wiederholt.
