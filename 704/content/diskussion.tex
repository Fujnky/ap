\section{Diskussion}
\label{sec:Diskussion}

Die nichtlinearen Fits der $\upgamma$-Absorptionskurven passen gut zu den Messwerten und in den logarithmierten Graphen ist ein linearer Zusammenhang erkennbar. $\mu$ und $N(0)$ können mit geringem Fehler bestimmt werden. Der Abschirmungskoeffizient von Kupfer liegt nur knapp unter dem theoretisch berechneten Compton-Koeffizienten, daher scheint der Compton-Effekt die entscheidende Rolle zu spielen. Für Blei ist der berechnete Wert deutlich geringer, also scheint der Photoeffekt einen signifikanten Anteil an der Absorption zu haben.

Auch bei der $\upbeta$-Strahlung passen die linearen Regressionen gut zu den Messdaten. Der Fehler ist akzeptabel und die berechnete Maximalenergie liegt in einem plausiblen Bereich.
