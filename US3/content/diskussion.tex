\section{Diskussion}
\label{sec:Diskussion}

Die Strömungsgeschwindigkeit kann im ersten Versuchsteil mit geringer Messabweichung (Im Mittel $\SI{8.1}{\percent}$) zwischen den verschiendenen Dopplerwinkeln ermittelt werden. Die Durchflussgeschwindigkeit ist wie erwartet bei kleineren Rohrdurchmessern (Gesetz von Bernoulli) und höherem Volumenstrom höher. Mögliche Fehlerquellen sind etwa Fertigungstoleranzen beim Ultraschallkopf oder die nur als allgemeiner Literaturwert bekannte Schallgeschwindigkeit von Acryl. Andererseits wird der Schlauch, durch den die Flüssigkeit nicht berücksichtigt, obwohl er potentiell auch eine andere Schallgeschwindigkeit besitzt.
Im Strömungsprofil kann zwischen \SI{33}{mm} und \SI{42}{mm}, also innerhalb des Rohres, ein annähernd laminarer Geschwindigkeitsverlauf gefunden werden. Außerhalb dieses Bereichs ist die Geschwindigkeit konstant, obwohl eigentlich gar keine Bewegung zu erwarten wäre. Dies hat vermutlich messtechnische Gründe. In diesem Bereich ist bei der Intensitätsverteilung ein linear steigender Verlauf zu beobachten.
