\section{Auswertung}
\label{sec:Auswertung}

\subsection{Strömungsgeschwindigkeit}

Gemäß \eqref{eqn:delta_f} lässt sich für die Strömungsgeschwindigkeit des Mediums die Formel
\begin{equation}
  v = \frac{\Delta f c_\text{L}}{2 f_0 \cos \alpha}
\end{equation}
aufstellen. Dabei ist $\alpha$ der Dopplerwinkel, welcher aus dem Prismenwinkel über die Beziehung
\begin{equation}
  \alpha = \frac{\pi}{2} - \arcsin\left({\sin \theta \cdot \frac{c_\text{L}}{c_\text{P}}}\right)
\end{equation}
ermittelt wird.

Die berechneten und gemittelten Geschwindigkeiten zu den verschiedenen Rohrduchmessern und Dopplerwinkeln finden sich in Tabelle~\ref{tab:erg}. Weiterhin wird gefordert, $\Delta f / \cos \alpha$ über $v$ aufzutragen, da jedoch zuvor $v$ über einen proportionalen Zusammenhang aus $\Delta f$ ermittelt wurde, ergeben sich lediglich Punkte auf einer Ursprungsgeraden. Die Diagramme finden sich in Abb.~\ref{fig:a0},~\ref{fig:a1} und \ref{fig:a2}.
\begin{table}
        \caption{Messergebnisse aus dem A-Scan. Neben den abgelesenen und berechneten Daten $d_n$ sind auch die zuvor mittels Messschieber bestimmten Abmessungen $d_n^\text{mech}$ eingetragen.}
        \centering
        \label{tab:a}
        \begin{tabular}{l@{}S[round-mode=off, table-format=2.0]S[table-format=2.3, round-precision=4, round-mode=off] S[table-format=2.2, round-precision=4, round-mode=figures] S[table-format=1.4, round-precision=4, round-mode=figures] S[table-format=2.3, round-precision=4, round-mode=figures] S[table-format=2.2, round-precision=4, round-mode=figures] S[table-format=2.4, round-precision=4, round-mode=off] } \toprule & {$\text{Störstelle}$}& {$d_1/\si{mm}$}& {$d_2/\si{mm}$}& {$2r/\si{mm}$}& {$d_1^\text{mech}/\si{mm}$}& {$d_2^\text{mech}/\si{mm}$}& {$2r^\text{mech}/\si{mm}$}\\\midrule
& 1 & 18.02 & 61.56149999999999522515 & 0.46050000000000257394 & 19.62000000000000099476 & 59.61999999999999744205 & 0.80 \\
& 2 & 19.93 & 59.78699999999999192823 & 0.32400000000000828138 & 17.76000000000000156319 & 61.29999999999999715783 & 0.98 \\
& 3 & 61.29 & 13.78650000000000019895 & 4.96500000000000429878 & 61.28000000000000113687 & 13.43999999999999950262 & 5.32 \\
& 4 & 54.05 & 22.11299999999999599254 & 3.87300000000000821387 & 53.84000000000000341061 & 21.69999999999999928946 & 4.50 \\
& 5 & 46.68 & 30.43950000000000244427 & 2.91749999999998976818 & 46.50000000000000000000 & 30.07999999999999829470 & 3.46 \\
& 6 & 39.04 & 39.03900000000000147793 & 1.96199999999999152855 & 38.60000000000000142109 & 38.71999999999999886313 & 2.72 \\
& 7 & 31.12 & 47.09249999999999403144 & 1.82550000000000767209 & 30.82000000000000028422 & 46.88000000000000255795 & 2.34 \\
& 8 & 23.07 & 55.14600000000000079581 & 1.82550000000000411937 & 22.78000000000000113687 & 54.82000000000000028422 & 2.44 \\
& 9 & 15.01 & 62.92650000000001142553 & 2.09849999999999115019 & 14.82000000000000028422 & 62.97999999999999687361 & 2.24 \\
& 10 & 7.23 & 70.98000000000000397904 & 1.82549999999999990052 & 6.87999999999999989342 & 70.79999999999999715783 & 2.36 \\
& 11 & 55.82 & 15.42450000000000009948 & 8.78699999999999548095 & 55.11999999999999744205 & 14.63000000000000078160 & 10.29 \\
 \bottomrule \end{tabular} \end{table}

\fig{build/a0.pdf}{Plot von $\Delta f / \cos \alpha$ über $v$ bei $d=\SI{16}{mm}$.}{a0}
\fig{build/a1.pdf}{Plot von $\Delta f / \cos \alpha$ über $v$ bei $d=\SI{10}{mm}$.}{a1}
\fig{build/a2.pdf}{Plot von $\Delta f / \cos \alpha$ über $v$ bei $d=\SI{7}{mm}$.}{a2}

\subsection{Strömungsprofil}
Im zweiten Versuchsteil wurde das Strömungsprofil im $\SI{10}{mm}$-Rohr durch Variation der Messtiefe ermittelt. In den Abbildungen~\ref{fig:b1} und \ref{fig:b2} sind Strömungsgeschwindigkeit bzw. Streuintensität über die Tiefe aufgetragen, jeweils für \SI{70}{\percent} und \SI{45}{\percent} Pumpleistung.
\fig{build/b1.pdf}{Plot der Strömungsgeschwindigkeit in Abhängigkeit von der Messtiefe.}{b1}
\fig{build/b2.pdf}{Plot der Streuintensität in Abhängigkeit von der Messtiefe.}{b2}

\newpage
\subsection{Messdaten}
\input{build/datena.tex}
\begin{table}
        \caption{Messdaten, Hallspannungsmessreihe.}
        \centering
        \label{datena}
        \begin{tabular}{
          l@{}
          S[round-mode=off, table-format=1.1]
          S[round-mode=off, table-format=3.0]
          S[round-mode=off, table-format=1.0]
          |
          S[round-mode=off, table-format=2.1]
          S[round-mode=off, table-format=3.0]
          S[round-mode=off, table-format=2.0]
        }
        \toprule
        & {$I/\si{A}$}& {$U_\text{Zn}/\si{µV}$}& {$U_\text{Cu}/\si{µV}$}& {$I/\si{A}$}& {$U_\text{Zn}/\si{µV}$}& {$U_\text{Cu}/\si{µV}$}\\
        \midrule
        &  0.0 & 0   & 0 & 5.5  & 428 & 7  \\
        &  0.5 & 41  & 1 & 6.0  & 468 & 8  \\
        &  1.0 & 82  & 2 & 6.5  & 503 & 9  \\
        &  1.5 & 115 & 2 & 7.0  & 545 & 10 \\
        &  2.0 & 155 & 3 & 7.5  & 579 & 10 \\
        &  2.5 & 195 & 4 & 8.0  & 620 & 11 \\
        &  3.0 & 235 & 4 & 8.5  & {}  & 12 \\
        &  3.5 & 270 & 5 & 9.0  & {}  & 12 \\
        &  4.0 & 308 & 6 & 9.5  & {}  & 13 \\
        &  4.5 & 348 & 6 & 10.0 & {}  & 14 \\
        &  5.0 & 386 & 7 &   {} & {}  & {} \\
 \bottomrule \end{tabular} \end{table}

